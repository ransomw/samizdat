\chapter{Topological Vector Spaces}

\section{Introduction}
Provisional definitions to be made precise in later sections

\section{Normed Spaces}
\begin{defn}
  a vector space, $X$, is a \emph{normed space} if \\
  ${\forall x\in X\exists \|x\|\in R\st}$
  \begin{enumerate}[label=(\alph*)]
  \item\label{defn:norm:1}\label{defn:norm:triangle}
    ${\|x+y\|\leq\|x\|+\|y\|,\,\forall x,y\in X}$
  \item\label{defn:norm:2}\label{defn:norm:scalar-mult}
    ${\|\alpha x\|=|\alpha|\|x\|,\,\forall x\in X\,\forall\text{scalars }\alpha}$
  \item\label{defn:norm:3}
    ${\forall x\in X,\, x\neq 0\Rightarrow \|x\|>0}$.
  \end{enumerate}
  Then, ${\|x\|}$ is called the \emph{norm} of $x$, and
  ``norm'' is also used to describe the function
  \[ \|\cdot\|: X\rightarrow R\quad x\mapsto\|x\| . \]
\end{defn}

\begin{rem}{}
  A normed space may be regarded as a metric space by
  defining distance
  \[ d(x, y) = \|x-y\| \]
\end{rem}
\begin{proof}\
  \begin{description}
  \item[(${0\leq d(x,y)<\infty,\,\forall x,y\in X}$)]
    STS ${\|x\|\geq 0,\,\forall x\in X}$.
    By \ref{defn:norm:3} for ${x\neq 0}$. \\
    For ${x=0}$,
    ${\alpha x = 0,\,\forall\text{scalars }\alpha}$,
    so ${x\neq 0\lightning\ref{defn:norm:scalar-mult}}$
    by cancellation, provided
    \[
    \exists\text{ scalars }\alpha_1,\alpha_2 \st |\alpha_1|\neq|\alpha_2|.
    \]
  \item[(${d(x,y)=0\iffa x=y}$)]
    \[
    x=y \iffa x - y = 0 \stackrel{\dag}{\iffa}
    \|x-y\|\stackrel{\text{def}}{=} d(x,y) = 0
    \]
    observing case ${x=0}$ of the previous part ($\dag$).
  \item[(${d(x,y)=d(y,x)}$)]
    \[ |-1| = |1|\land\ref{defn:norm:scalar-mult} \]
  \item[($\triangle$ inequality)]
    \[
    \|x-z\| = \|(x-y)+(y-z)\|\stackrel{\ref{defn:norm:triangle}}{\leq}
    \|x-y\|+\|y-z\|
    \]
  \end{description}
\end{proof}

\begin{defn}\
  \begin{itemize}
  \item
    the \emph{open ball} with radius $r$, center $x$
    in any metric space is
    \[ B_r(x) = \{y\mid d(x, y)<r\} \]
  \item
    the \emph{open unit ball} and \emph{closed unit ball}
    of a normed space, $X$, are
    \[
    B_1(0) = \{x\in X\mid \|x\|<1\}
    \quad\text{and}\quad
    \overline{B_1(0)} = \{x\in X\mid \|x\|\leq1\},
    \]
    respectively.
  \end{itemize}
\end{defn}

\begin{rem}{}
  a topology may be obtained on a metric space by defining open sets
  to be (possibly empty) unions of open balls
\end{rem}
\begin{proof}\
  \begin{description}
  \item[(finite $\cap$)]
    fix ${U, V\in\tau}$.
    ${U\cap V}$ the union of open balls is implied by
    \[
    \forall x\in U\cap V
    \exists\text{open ball } B\subset U\cap V
    \st x\in B
    \]
    --- i.e. representation as balls needn't be unique.

    So the usual, formally weaker, statement
    \[
    \forall x\in U\cap V\exists r>0
    \st B_r(x)\subset U\cap V
    \]
    suffices, and for this,
    \[
    U, V\in\tau \Longrightarrow
    \exists B_1\subset U, B_2\subset V
    \st x\in B_1\land x\in B_2
    \]
    \sketch{%
      choose $\mathtt{r}$ based on $\mathtt{x}$
      and centers of $\mathtt{B_1, B_2}$ using $\mathtt{\triangle}$}
  \end{description}
\end{proof}

\begin{rem}{}
  for normed space, vector addtion and scalar multiplication
  are continuous in the topology defined by open balls
  \sketch{clarify statement and prove}
\end{rem}

\begin{defn}\
  \begin{itemize}
  \item
    a metric space is \emph{complete} if Cauchy sequences converge
  \item
    a \emph{Banach space} is a normed space
    that's complete wrt. the metric defined by the norm
  \end{itemize}
\end{defn}

\section{Examples of function spaces that aren't (necessarily) Banach spaces}

\begin{enumerate}[label=(\alph*)]
\item
  continuous $\mathbb{C}$-valued functions on open ${\subset R^n}$
\item
  holomorphic functions on open ${\subset\mathbb{C}}$
\item
  $C^\infty$ $\mathbb{C}$-valued functions on $R^n$ with compact
  support that (the support) has ${\neq\emptyset}$ interior
\item
  some spaces used in distribution theory
\end{enumerate}

These spaces all have topologies that can't be induced by norms.

\paragraph{}
{\Large
  $\Uparrow$\hspace{\stretch{1}}
  All preceeding \S 's considered motivation
  \hspace{\stretch{1}}$\Uparrow$
}

\pagebreak

\section{Vector spaces}
For this \S, let $\Phi$ denote either $R$ or $\mathbb{C}$.

\begin{defn}\
  \begin{itemize}
  \item
    a \emph{scalar} is an element of the \emph{scalar field} $\Phi$
  \item
    a \emph{vector space over} $\Phi$ is a set $X$
    with elements called \emph{vectors}, together with two operations,
    \emph{addition} and \emph{scalar multiplication},
    \[
    +: X^2\rightarrow X
    \quad\text{and}\quad
    \cdot: X^2\rightarrow X
    \qquad\st
    \]
    \begin{enumerate}[label=(\alph*)]
    \item
      ${\forall x,y\in X,\ x+y=y+x\land x+(y+z)=(x+y)+z}$,\\
      ${\exists! 0\in X\st\forall x\in X,\ x+0=x}$, and\\
      ${\forall x\in X\exists! -x\in X\st x+(-x)=0}$, \\
      where $0$ above is called the \emph{zero vector}
      or \emph{origin} of $X$.
    \item
      ${\forall\alpha,\beta\in\Phi,\ \forall x,y\in X}$\\
      ${1x=x}$, ${\alpha(\beta x)=(\alpha\beta)x}$, and \\
      ${\alpha(x+y)=\alpha x+\alpha y\land (\alpha+\beta)x=\alpha x+\beta y}$.
    \end{enumerate}
  \end{itemize}
\end{defn}

\begin{rem}{B}
  there are distinct ${0\in\Phi}$ and ${0\in X}$
\end{rem}

\begin{defn}
  a \emph{real} (resp. \emph{complex}) \emph{vector space}
  is one where ${\Phi=R}$ (resp. ${=\mathbb{C}}$)

  and throughout the text, vector space is used to mean one of these
  two cases
\end{defn}

\begin{addcomment}
  a vector space is a module where the ring is a field.
\end{addcomment}

\begin{defn}\
  \begin{itemize}
  \item
    for vector space $X$ over $\Phi$, ${A\subset X}$, ${B\subset X}$,
    ${x\in X}$, and ${\lambda\in\Phi}$, put
    \begin{IEEEeqnarray*}{rCl}
      x+A=\{x+a\mid a\in A\}, & \quad & x-A=\{x-a\mid a\in A\}, \\
      A+B=\{a+b\mid a\in A\land b\in B\}, & \quad &
      \lambda A=\{\lambda a\mid a\in A\},
    \end{IEEEeqnarray*}
    and in particular ${-A}$ is the set of inverses of elements of $A$.
  \item
    for vector space $X$ over $\Phi$, ${Y\subset X}$ is a
    \emph{subspace} of $X$ if $\restr{+}{Y^2}$ and
    $\restr{\cdot}{\Phi\times Y}$ satisfy the definition
    of a vector space for $Y$.
    \begin{addcomment}
      usually proof of the form
      \[
      \forall \alpha,\beta\in\Phi,\forall x,y\in Y,\,
      \alpha x+\beta y\in Y
      \]
      suffices.  see \ref{rem:01:vecsubspc-char} below.
    \end{addcomment}
  \end{itemize}
\end{defn}

\begin{rem}{B}\label{rem:01:vecsubspc-char}
  \[
  Y\subset X\text{ a subspace }
  \iff
  0\in Y\land \forall\alpha,\beta\in\Phi,\, \alpha Y+\beta Y\subset Y
  \]
\end{rem}
\begin{proof}
  \sketch{todo (not in text)}
\end{proof}

\begin{defn}
  for a vector space $X$,
  \begin{itemize}
  \item
    a set ${C\subset X}$ is \emph{convex} if
    \[ \forall t\in[0,1],\, tC+(1-t)C\subset C \]
    or equivalently
    \[ \forall t\in[0,1],\forall x,y\in C,\, tx+(1-t)y\in C . \]
  \item
    a set ${B\subset X}$ is \emph{balanced} if
    \[
    \forall\alpha\in\Phi,\,
    \|\alpha\|\leq 1\Rightarrow \alpha B\subset B
    \]
  \item
    ${\{u_1,\ldots,u_n\}\subset X}$ is a \emph{basis} for $X$ if
    \[
    \forall x\in X \exists! (\alpha_1,\ldots,\alpha_n)\in\Phi^n
    \st x = \sum_1^n\alpha_j u_j
    \]
  \item
    $X$ has \emph{dimension} ${n\in\mathbb{N}}$, written ${\dim X=n}$,
    if it has a basis of $n$ elements,
    and is then said to have \emph{finite dimension}
    \sketch{?
      what facts about basis and $\mathtt{\dim}$ are implicit here?\\
      what (e.g. existence) can be added?}
  \end{itemize}
\end{defn}

\begin{exa}
  for ${X=\mathbb{C}}$ over ${\Phi=\mathbb{C}}$,
  the balanced sets are $\mathbb{C}$, $\emptyset$,
  and the discs centered at $0$,\\
  while for ${X=R^2}$, ${\Phi=R}$, there are more balanced sets,
  including line segements with midpoints at ${(0,0)}$
\end{exa}

\sketch{\loopy
  what're images of line segments in $\mathbb{C}$ under
  scalar mult?}

\section{Topological spaces}

\begin{defn}\
  \begin{itemize}
  \item
    a \emph{topological space} is ${(S,\tau)\st}$
    \sketch{the usual}
  \item
    for ${E\subset S}$
    \[
    \overline E = \cap\{C\supset E\mid C\text{ closed}\}
    \quad\text{and}\quad
    E^\circ = \cup\{U\subset E\mid U\text{ open}\}
    \]
    are the \emph{closure} and \emph{interior} of $E$
  \item
    a \emph{neighborhood} of a point ${p\in S}$ is
    ${E\in\tau\st p\in E}$
  \item
    $S$ is \emph{Hausdorff} if distinct points have
    disjoint neighborhoods
  \item
    \emph{compact}
    \sketch{the usual: open cover has finite subcover}
  \item
    ${\tau'\subset\tau}$ is a \emph{base} for $\tau$ if
    \[ \forall E\in\tau\exists\mathcal{E}\in\tau'\st\cup\mathcal{E}=E \]
  \item
    ${\gamma\subset\tau}$ is a \emph{local base} at ${p\in S}$ if
    \begin{itemize}
    \item ${\forall E\in\gamma,\, p\in E}$\\
      (i.e. --- $\gamma$ consists of neighborhoods of $p$), and
    \item ${\forall E\in\tau,\,p\in E\Rightarrow\exists F\in\gamma\st F\subset E}$\\
      (i.e. --- ``every neighborhood of $p$ contains a member
      of $\gamma$).
    \end{itemize}
  \item
    \emph{inherited topology} on ${E\subset S}$
    \sketch{the usual}
  \item
    topologies and metrics thought of as \emph{compatible},
    a bi-directional word, when open balls
    (from the metric) `generate'
    (i.e. --- specify the strongest/smallest topology containing)
    of ``induce'' the topology.
    \sketch{could be cleaner}
  \item
    for Hausdorff $X$, a sequence ${\{x_n\}\subset X}$ \emph{converges}
    to ${x\in X}$, written ${\lim_{n\rightarrow\infty} x_n = 0}$, if
    \[ \{x_n\}\cap(X\setminus N)\text{ finite} , \forall N \nbhd x \]
    \sketch{?
      why specify $X$ Hausdorff?}
  \end{itemize}
\end{defn}

\section{Topological Vector Spaces}

\begin{defn}
  a \emph{toplogical vector space} ${(X,\tau)}$ is a
  $\Phi$-vector space, $X$, with a topology, $\tau$,$\st$
  \begin{enumerate}[label=(\alph*)]
  \item\label{defn:01:tvs:pts-cl}
    \[ \setcomp{\{p\}}\in\tau,\,\forall p\in X \]
    (i.e. --- every point is a closed set)
  \item
    \[
    X^2\rightarrow X \quad (x,y)\mapsto x+y
    \qquad\text{and}\qquad
    \Phi\times X\rightarrow X \quad (\alpha, x)\mapsto \alpha x
    \]
    are both continuous with product topologies on
    $X^2$ and ${\Phi\times X}$, resp.
  \end{enumerate}
  and $\tau$ is called a \emph{vector topology}
\end{defn}

\begin{rem}{}\
  \begin{itemize}
  \item
    some developments omit \ref{defn:01:tvs:pts-cl} from
    definition and repeatedly include in hypothesis
  \item
    since this text doesn't include a definition of product topologies,
    it includes specific (metric space) definitions of continuity:
    \begin{itemize}
    \item
      ${X^2\rightarrow X \quad (x,y)\mapsto x+y}$ continuous if
      \[
      \forall x,y\in X\,\forall U\nbhd x + y\,
      \exists V\nbhd x, W\nbhd y
      \st V + W \subset U
      \]
    \item
      ${\Phi\times X\rightarrow X}$ continuous if
      \[
      \forall x\in X, \alpha\in\Phi, U\nbhd \alpha x\,
      \exists r>0, V\nbhd x
      \st
      |\beta-\alpha|<r\Rightarrow \beta W\subset V
      \]
    \end{itemize}
  \end{itemize}
\end{rem}

\begin{defn}
  a subset, $B$, of a topological vector space, $X$,
  is \emph{bounded} if
  \[
  \forall V\nbhd 0\in X\exists s>0
  \quad\st\quad
  \forall t\in R,\, t>s\Rightarrow B\subset tV
  \]
\end{defn}

\section{Invariance}
\begin{defn}
  for topological $\Phi$-vector space, $X$, and any ${\alpha\in X}$,
  ${\lambda\in\Phi}$, the \emph{translation operator}
  ${T_\alpha:X\rightarrow X}$ and \emph{multiplication operator}
  ${M_\lambda: X\rightarrow X}$ are ${T_\alpha(x) = \alpha + x}$
  and ${M_\lambda(x)=\lambda x}$.
\end{defn}

\begin{prop}\label{prop:01:trans-mult-are-homeo}
  $T_\alpha$ and $M_\lambda$ are \emph{homeomorphisms} ${X\rightarrow X}$
  (i.e. continuous bijections with continuous inverses)
\end{prop}
\begin{proof}
  \sketch{
    todo:  in text but sketchy}
\end{proof}

\begin{defn}
  a topology $\tau$ on a vector space $X$
  (not necessarily with a topology)
  is \emph{translation-invariant} (or \emph{invariant} when
  context is clear) $\iff$
  \[
  \forall a\in X\,\forall E\subset X,\,
  E\in\tau \Leftrightarrow a+E\in\tau
  \]
\end{defn}

\begin{rem}{A}\label{rem:01:tvs-trans-invar}
  by \ref{prop:01:trans-mult-are-homeo}, every \emph{vector topology}
  (i.e. a topology on a vector space that satsfies the definition
  of a topological vector space) is invariant
\end{rem}

\begin{defn}
  for a topological vector space, $X$, a \emph{local base}
  is a local base (as defined for arbitrary topological spaces)
  at ${0\in X}$
\end{defn}

\begin{rem}{}
  if $\mathcal{B}$ is a local base of a topological vector space,
  ${(X,\tau)}$, then
  \begin{IEEEeqnarray*}{lr}
    \forall E\subset X, E\in\tau\iff & \\
    \exists S\subset\mathscr{P}(X)\st & \\
    & E =\cup S\land\big(
    F\in S\Leftrightarrow \exists G\in\mathcal{B}\,\exists a\in X
    [F=a+G]\big)
  \end{IEEEeqnarray*}
  (i.e. --- open sets are precisely unions of translates of
  $\mathcal{B}$).
\end{rem}

\begin{defn}
  a metric, $d$, on a vector space, $X$, is \emph{invariant} if
  \[d(x+z,y+z)=d(x,y),\forall x,y,z\in X \]
\end{defn}

\section{Types of topological vector spaces}
Throughout this section, $X$ a topological vector space with
a topology, $\tau$.

\begin{defn}\
  \begin{itemize}
  \item $X$ is \emph{locally convex}
    if there's a local base $\mathcal{B}$ $\st$ $E$ convex
    ${\forall E\in\mathcal{B}}$
  \item $X$ is \emph{locally bounded}
    if ${0\in X}$ has a bounded neighborhood
  \item $X$ is \emph{locally compact} if
    ${\exists E \nbhd 0 \st \overline{E}\text{ compact}}$
  \item $X$ is \emph{metrizable} if $\tau$ compatible with some
    metric $d$
  \item $X$ is an \emph{F-space} if $\tau$ compatible with
    a complete, invariant metric $d$
  \item a \emph{Frechet space} is
    a locally convex F-space
  \end{itemize}
\end{defn}

\begin{rem}{}
  F-space/Frechet space distinction is not the same in all
  texts
\end{rem}

\begin{defn}
  \sketch{
    todo: precise defn. of normed space and Banach space,
    following motivation in 1.2 \emph{of Rudin's text}
  }
\end{defn}

\begin{defn}
  $X$ has the \emph{Heine-Borel property} if
  \[
  \forall E\subset X,
  E\text{ closed}\land E\text{ bounded}
  \Rightarrow E\text{ compact}
  \]
\end{defn}

\section{Separation properties}

\begin{defn}
  a subset $U$ of a topological vector space $X$ is
  \emph{symmetric} if ${U=-U}$.
\end{defn}

\begin{prop}\label{prop:01:A}\label{prop:01:symmetric-subnbhd}
  for any topological vector space $X$,
  \[
  \forall W\nbhd 0,\, \exists U\nbhd 0\st
  U=-U\land U+U\subset W
  \]
  (i.e. --- every neighborhood of zero contains a symmetric
  neighborhood of zero)
\end{prop}
\begin{proof}
  addition is continuous
  ${X^2\rightarrow X}$ and ${(+)(0,0)\mapsto 0}$, so\\
  \[{(0,0)\in\text{open }{(+)}^{-1}(W)\subset X^2}.\]
  \paragraph{}
  By Munkres~\ref{munkres-thm:15:fin-prod-subbasis},
  \begin{description}
  \item{fact:} the set of all finite intersections of preimages of
    open sets under coordinate maps is a base for the product
    topology
  \end{description}

  So choose such a basis element that contains the point $(0,0)$
  and is included in ${(+)}^{-1}(W)$:
  Fix
  \[V_1, V_2\in \tau_X
  \st
  (0,0)\in V_1\times V_2\subset {(+)}^{-1}(V)\]
  --- i.e. such that ${V_1+V_2\subset W}$.

  Set ${U=V_1\cap V_2\cap(-V_1)\cap(-V_2)}$.
  Then
  ${0\in V_1,V_2\Rightarrow V_1,V_2\subset W \Rightarrow U\subset W}$,\\
  and
  ${x\in U\Rightarrow x\in V_1\cap V_2\Rightarrow -x\in (-V_1)\cap (-V_2)\supset U}$,\\
  and
  ${-x\in U\Rightarrow-x\in (-V_1)\cap(-V_2)\Rightarrow x\in V_1\cap V_2\supset U}$.
\end{proof}


\begin{thm}\label{thm:01:1.10}\label{thm:01:cpt-cl-sep}
  for all topological vector spaces, $X$, and all ${K, C\subset X}$,
  \begin{IEEEeqnarray*}{c}
    K\text{ compact}\land C\text{ closed}\land K\cap C = \emptyset
    \Rightarrow \\
    \exists V\nbhd 0\st (K+V)\cap(C+V)=\emptyset
  \end{IEEEeqnarray*}
\end{thm}
\begin{proof}
  \[K=\emptyset \Rightarrow K+V = \emptyset, \forall V\in\tau_X\]
  so consider $K\neq\emptyset$.
  Fix any ${x\in K\Rightarrow x\not\in C}$,
  and let $T_x: X\rightarrow X$ be the translation $x\mapsto 0$.

  Apply \ref{prop:01:symmetric-subnbhd} to $T_x(\setcomp{C})$,
  to obtain witness $\tilde U$, and then apply it again
  to $\tilde U$ to obtain
  \[ \text{symmetric } V_x\nbhd 0
  \st V_x + V_x + V_x + V_x \cap T_x C = \emptyset \]
  and more specifically,
  \[ 0 + V_x + V_x + V_x \cap T_x C = \emptyset
  \Leftrightarrow
  x + V_x + V_x + V_x \cap C = \emptyset . \]
  Which, together with $V_x$ symmetric implies
  \sketch{consider logical negation of previous
    --- i.e. $\exists a,b,c \in V_x, p\in C$
    such that ${x+a+b+c = p}$ and rename $c=-c$,
    say, since $V_x=-V_x$.
  }
  \begin{equation}\label{eq:01:thm:1.10:1}
    x + V_x + V_x \cap (C + V_x) = \emptyset.
  \end{equation}

  Now since $x\in K$ arbitrary and
  ${\cup_{x\in K} x + V_x\supset K}$ is an open cover of $K$,
  choose finite subcover $\{x_j + V_{x_j}\}_1^n$, and put
  ${V:=\cap_1^nV_{x_j}}$.  Then,
  \[K + V \subset
  \left(\cup_1^n (x_j + V_{x_j})\right) + V
  \stackrel{\dag}{\subset}
  \left(\cup_1^n (x_j + V_{x_j} + V)\right) \subset
  \left(\cup_1^n (x_j + V_{x_j} + V_{x_j})\right)
  \]
  where ($\dag$) follows from considering point-wise representations
  of elements in each set.  Use \ref{eq:01:thm:1.10:1} to conclude.

\end{proof}

\begin{cor}
  disjoin neighborhoods contain $K$ and $C$
\end{cor}
\begin{proof}
  \[ A+V = \cup\{T_aV\mid a\in A\} \]
\end{proof}

\begin{thm}
  $\forall$ topological vector spaces $X$,
  $\forall$ local bases $\mathscr B$ of $X$
  (i.e. --- local bases in the sense of general topology
  at the origin),
  $\forall$ $W\in\mathscr B$
  \[ \exists U\in\mathscr B \st \overline U \subset W .\]
\end{thm}
\begin{proof}
  $\{0\}$, like any finite set, is compact,
  so put $K=\{0\}$, $C=\setcomp{W}$,
  and choose $V$ satisfying Theorem~\ref{thm:01:cpt-cl-sep}.

  For any $C, V, K$ as in Theorem~\ref{thm:01:cpt-cl-sep},
  \[
  C+V\text{ open }\land (C+V)\cap(K+V)=\emptyset
  \Rightarrow
  (C+V)\cap\overline{K+V}=\emptyset,
  \]
  since $K+V\subset\setcomp{C+V}$ and the closure is the
  intersection of all such sets containing $K+V$.

  Now, returning the the context of this theorem,
  $\exists U\subset\mathscr B\st U\subset V$
  by definition of local base, and
  $C\subset C+V\Rightarrow \setcomp{C}\supset\setcomp{(C+V)}$,
  $U\subset V\Rightarrow \overline{U} \subset \overline{V}$,
  so conclude
  \[
  \overline{U} \subset \overline{V}
  \subset \overline{K+V} \subset \setcomp{(C+V)}
  \subset \setcomp{C} = W.
  \]
  \sketch{
    this is a direct application of \ref{thm:01:cpt-cl-sep}.
    the separation discussed here is
    the additional separation provided by that theorem between
    $\{0\}$ and the compliment of any element of a local base.
  }
\end{proof}

\begin{rem}{}
  none of the results thus far have used that each point
  in a topological vector space is closed.
\end{rem}

\begin{thm}\label{thm:01:tvs-haus}
  topological vector spaces are Hausdorff
\end{thm}
\begin{proof}
  just about immediate from Theorem~\ref{thm:01:cpt-cl-sep},
  its corollary,
  and the defintion of topological vector spaces:

  For any $p, q\in X$, put $K=\{p\}$, $C=\{q\}$.
  $K$ cpt as a finite set, $C$ closed by definition
  of topological vector space.
  Now the corollary is the definition of Hausdorff.
\end{proof}


\begin{rem}{C}\label{rem:01:closure-open-char}
  \[
  p \in \overline{E} \iff
  \forall U \nbhd p, U\cap E\neq\emptyset
  \]
\end{rem}
\begin{proof}
  \sketch{
    contrapositives in both directions by definition of closure.
  }
\end{proof}

\begin{lem}\label{lem:01:scalar-mult-intersection}
  for any topological vector space $X$ over scalar field $\Phi$
  \[
  \beta(A\cap B) = (\beta A)\cap(\beta B),\quad
  \forall A, B\subset X, \forall \beta\in\Phi
  \]
\end{lem}
\begin{proof}
  For any $q\in X$,
  \begin{IEEEeqnarray*}{lCr}
    q\in \beta(A\cap B)
    & \iff
    \exists a\in A \exists b\in B &
    \st q = \beta a = \beta b \\
    & \stackrel{\dag}{\Leftrightarrow}
    \exists v \in A\cap B &
    \st q = \beta v \\
    && \iff q \in \beta(A\cap B)
  \end{IEEEeqnarray*}
  where the forward direction of ($\dag$)
  follows from the cancellation
  \[ \beta a = \beta b \Rightarrow a = b, \]
  and is the only reason this proof of this lemma might not be
  totally silly to write down in the first place.
\end{proof}

\begin{lem}\label{lem:01:closure-scalar-mult}
  for any topological vector space $X$ over scalar field $\Phi$
  \[
  \alpha\overline{Y} = \overline{\alpha Y},\quad
  \forall Y\subset X, \forall \alpha\in\Phi
  \]
\end{lem}
\begin{proof}
  In case ${\alpha=0}$,
  both sides of the equality in the hypothesis are $\{0\}$,
  so choose any ${p\in X}$, ${\alpha\in\Phi}$, and ${\alpha\neq 0}$.
  \begin{IEEEeqnarray*}{lCr}
    p\in \overline{\alpha Y}
    & \stackrel{\dag}{\Leftrightarrow} &
    U\cap (\alpha Y) \neq\emptyset,\quad
    \forall U\nbhd p \\
    & \stackrel{\diamondsuit}{\Leftrightarrow}  &
    (\alpha^{-1} U)\cap Y \neq\emptyset,\quad
    \forall U\nbhd p \\
    & \stackrel{\heartsuit}{\Leftrightarrow}  &
    V\cap Y \neq\emptyset,\quad
    \forall V\nbhd \alpha^{-1}p \\
    & \stackrel{\dag}{\Leftrightarrow} &
    \alpha^{-1}p \in \overline{Y} \iff p\in\alpha\overline{Y}
  \end{IEEEeqnarray*}
  where
  ($\dag$) follows from Remark~\ref{rem:01:closure-open-char},
  ($\diamondsuit$) from Lemma~\ref{lem:01:scalar-mult-intersection},
  and
  ($\heartsuit$) from Proposition~\ref{prop:01:trans-mult-are-homeo}
  (i.e. --- ${x\mapsto\alpha x}$ is a homeomorphism).
  \sketch{$\loopy$
    ??? prove with intesection definition rather than limit-point
    characterizaton?}
\end{proof}

\begin{thm}\label{thm:01:tvs-char-00}
  for topological vector space $X$ over scalar field $\Phi$
  \begin{enumerate}[label=(\alph*)]
  \item\label{thm:01:tvs-char-00:1}
    \[
    \forall A\subset X, \overline{A} =
    \cap\{A+V\mid V\nbhd 0\in X\}
    \]
  \item\label{thm:01:tvs-char-00:2}
    \[
    \forall A, B\subset X,
    \overline{A}+\overline{B} \subset \overline{A + B}
    \]
  \item\label{thm:01:tvs-char-00:3}
    \[
    Y\subspc X \Rightarrow \overline{Y}\subspc X
    \]
  \item\label{thm:01:tvs-char-00:4}
    \[
    C\subset X \text{ convex}
    \Rightarrow
    \overline{C}, \setint{C} \text{ convex}
    \]
  \item\label{thm:01:tvs-char-00:5}
    \[
    B\subset X\text{ balanced} \Rightarrow
    \overline{B}\text{ balanced}
    \]
    and
    \[
    \forall B\subset X,
    B\text{ balanced}\land 0\in\setint{B}\Rightarrow
    \setint{B}\text{ balanced}
    \]
  \item\label{thm:01:tvs-char-00:6}
    \[
    E\subset X\text{ bounded} \Rightarrow
    \overline{E} \text{ bounded}
    \]
  \end{enumerate}
\end{thm}
\begin{proof}
  \begin{description}
  \item[\ref{thm:01:tvs-char-00:1}]
    \begin{IEEEeqnarray*}{lCr}
      x\in \overline{A} & \stackrel{\dag}{\Leftrightarrow} &
      (x+V)\cap A\neq\emptyset, \forall V\nbhd 0 \\
      & \iff & \forall V\nbhd 0 \exists p\in V\exists a\in A
      \st x + p =a \\
      & \iff & \forall V\nbhd 0, x \in A - V.
    \end{IEEEeqnarray*}
    where ($\dag$) follows from
    Remark~\ref{rem:01:closure-open-char} and
    the invariance of topology under translation in
    Remark~\ref{rem:01:tvs-trans-invar}.
    \paragraph{}
    Conclude by observing
    \[
    \{ V\subset X\mid V\nbhd 0\} =
    \{ -V\subset X\mid V\nbhd 0\}.
    \]
  \item[\ref{thm:01:tvs-char-00:2}]
    Choose any ${A,B\subset X}$,
    ${a\in\overline{A}, b\in\overline{B}}$,
    and ${W\nbhd a+b}$.
    \paragraph{}
    By Theorem~\ref{munkres-thm:15:fin-prod-basis} in Munkres
    and continuity of addition, choose
    \[ W_1 \nbhd a, W_2 \nbhd b \st W_1 + W_2 \subset W. \]
    Remark~\ref{rem:01:closure-open-char} and choice of ${a,b}$
    imply
    \[ A\cap W_1 \neq\emptyset \land B\cap W_2 \neq\emptyset, \]
    so choose any ${x\in A\cap W_1, y\in B\cap W_2}$ such that
    \[ x+y\in (A+B) \cap W \neq\emptyset, \]
    proving since ${a, b, W}$ arbitrary.
  \item[\ref{thm:01:tvs-char-00:3}]
    Choose any ${\alpha,\beta\in\Phi}$ and ${Y\subspc X}$.  Now,
    \[
    \alpha\overline{Y} + \beta\overline{Y}
    \stackrel{\dag}{=}
    \overline{\alpha Y} + \overline{\beta Y}
    \stackrel{\clubsuit}{\subset}
    \overline{\alpha Y + \beta Y}
    \stackrel{\spadesuit}{\subset} \overline{Y}
    \]
    where
    ($\dag$) follows from Lemma~\ref{lem:01:closure-scalar-mult},
    ($\clubsuit$) from part~\ref{thm:01:tvs-char-00:2} of this theorem,
    ($\spadesuit$) from
    Remark~\ref{rem:01:vecsubspc-char}, the sufficient condition
    for subspaces of a topological vector space,
    and the observation that
    ${A\subset B\Rightarrow \overline{A}\subset \overline{B}}$.
    \sketch{$\loopy$
      find or extract this last observation to a lemma somewhere,
      probably in Munkres}
    So Remark~\ref{rem:01:vecsubspc-char}
    is satisfied for $\overline{Y}$ as well.
  \item[\ref{thm:01:tvs-char-00:4}]
    \begin{description}
    \item[($\overline{C}$ convex)]
      Analogous to part~\ref{thm:01:tvs-char-00:3} of this theorem,
      \[
      \forall t\in[0, 1],\quad
      t\overline{C} + (1-t)\overline{C} \subset
      \overline{tC + (1-t)C}
      \stackrel{\spadesuit}{\subset} \overline{C}
      \]
      except use definition of convex sets rather than
      definition of closure in $\spadesuit$.
    \item[($\setint{C}$ convex)]
      \sketch{$\attn$ todo}
    \end{description}
  \item[\ref{thm:01:tvs-char-00:5}]
    \begin{description}
    \item[($\overline{B}$ balanced)]
      Analogous to part~\ref{thm:01:tvs-char-00:3} of this theorem,
      \[
      \forall \alpha\in\Phi,\quad
      \alpha\overline{B}\subset \overline{\alpha B}
      \stackrel{\spadesuit}{\subset} \overline{B}
      \]
      except use definition of balanced sets rather than
      definition of closure in $\spadesuit$.
    \item[($\setint{B}$ balanced)]
      \sketch{$\attn$ todo}
    \end{description}
  \item[\ref{thm:01:tvs-char-00:6}]
    \sketch{$\attn$ todo}
  \end{description}
\end{proof}
