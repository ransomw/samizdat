\chapter{Smooth Manifolds}

\section{Topological Manifolds}
\begin{defn}\
  \begin{itemize}
  \item
    a topological space, $M$, is a \emph{topological manifold} if
    \begin{enumerate}[label=(\roman*)]
    \item\label{defn:tm:haus} $M$ \emph{Hausdorff} --- i.e.
      \[
      \forall p, q\in M \exists U,V \in \topo_M \st
      U\cap V = \emptyset \land p\in U \land q\in V
      \]
    \item\label{defn:tm:secondcount}
      $M$ \emph{second countable} --- i.e.
      $\exists\mathcal E\subset \mathscr P(M)$,
      a countable basis for $\topo_M$
    \item\label{defn:tm:leuclid}
      $M$ is \emph{locally Euclidean of dimension} $n$ --- i.e.
      \begin{IEEEeqnarray*}{rCl}
        \forall p\in M
        \exists \varphi: U\subset M\rightarrow \widetilde U\subset R^n
        & \st & \\
        U\nbhd p\land
        \widetilde U\in \topo_M
        & \land & \varphi \text{ a homeomorphism},
      \end{IEEEeqnarray*}
      (recalling that a \emph{homeomorphism} is a continuous
      bijection with a continuous inverse).
    \end{enumerate}
  \item
    $\dim M = n$ or $M^n$ are used to indicate the dimension of
    a manifold, depending on context
  \end{itemize}
\end{defn}

\begin{rem}{A}
  for Hausdorff topological space, $X$
  \begin{enumerate}[label=(\alph*)]
  \item
    $\setcomp{\{p\}}\in\topo_X, \forall p\in X$
    (i.e. single point sets are closed)
  \item
    \sketch{``limits of convergent sequences unique'' (???)}
  \end{enumerate}
\end{rem}
\begin{proof}
  \sketch{exercise A.5}
\end{proof}

\begin{rem}{B}\label{rem:01:open-is-submani}
  any open subset a topological $n$-manifold is itself a
  topological $n$-manifold.
\end{rem}
\begin{proof}
  \sketch{
    $\attn$ todo --- apply A.5
  }
\end{proof}

\begin{rem}{C}
  $\emptyset$ a topological $n$-manifold $\forall n\in\mathbb N$
\end{rem}
\begin{proof}
  direct from definition
\end{proof}

\section{Coordinate charts}
Let $M$ be a topological $n$-manifold throughout this section.
\begin{defn}\
  \begin{itemize}
  \item
    a \emph{coordinate chart} (or \emph{chart}) is $(U,\varphi)$
    where $U\subset M$ open and
    $\varphi: U\rightarrow \widetilde U$
    is a homeomorphism to open $\widetilde U\subset R^n$.
    (as from definition of a topological $n$-manifold).
  \item
    a chart $(U,\varphi)$ is \emph{centered} at $p\in M$
    if $\varphi(p)=0$.
  \item
    $U$ is the \emph{coordinate domain} or
    \emph{coordinate neighborhood} of its points for any chart
    $(U,\varphi)$,
    and if $\varphi(U)\subset R^n$ is an open ball, then
    $U$ is a \emph{coordinate ball}.
  \item
    a chart's map, $\varphi$, is called a
    \emph{(local) coordinate map}, and
    $\{x^i: U\rightarrow R\}_{i=1}^n$
    defined by
    $x^i=\pi_i\circ\varphi$
    such that
    $\varphi(p)=(x^1\small(p\small),\ldots,x^n\small(p\small))$
    are called \emph{local coordinates} on $U$,
    sometimes denoting charts a-la
    $(U, \small(x^i\small))$.
  \item
    the phrase
    ``$(U,\varphi)$ a chart containing $p$''
    is sometimes used to abbreviate
    ``$(U,\varphi)$ a chart such that $p\in U$''
  \end{itemize}
\end{defn}

\sketch{
  skipping several examples for now,
  projective space and the product manifold in particular
}

\section{Topological properties of manifolds}
\begin{defn}
  for a topological space, $X$, a subset $K\subset X$ is
  \emph{precompact} (or \emph{relatively compact}) in $X$ if
  $\overline K$, its closure in $X$, is compact
\end{defn}

\begin{lem}\label{lem:1.6}\label{lem:01:tm-basis}
  every topological manifold has a countable basis
  of precompact coordinate balls.
\end{lem}
\begin{proof}\
  \begin{itemize}
  \item[(Special case)]
    Fix
    \[ M^n\st \exists\text{ chart } (M,\varphi). \]
    Put
    \[
    \mathscr B = \{B_r(x)\subset R^n\mid
    r\in\mathbb Q\land \pi_i(x)\in\mathbb Q, \forall i\in[n] \land
    \overline{B_r(x)}\subset\varphi(M) \}.
    \]
    Now
    \[ B \text{ precompact in } \varphi(M), \forall B\in\mathscr B, \]
    \sketch{\attn todo: check.}
    and $\mathscr B$ countable as the union of countable sets.
    Moreover, and also, $\mathscr B$ a basis for
    $\topo_{\varphi(M)}$.
    \sketch{\attn todo: check.}
    Since $\varphi$ continuous,
    $\{\varphi^{-1}(B)\subset M\mid B\in\mathscr B\}$
    is the desired basis for $\topo_M$.
  \item[(General case)]
    Fix arbitrary $M^n$.
    By definition of a topological manifold,
    \[
    \forall p\in M\exists\text{ chart }(U_p, \varphi_p)\st p\in U_p.
    \]
    Hence $\{U_p\}_{p\in M}$ covers $M$,
    and since $M$ second countable by definition, by
    \begin{fact}
      every open cover of a second countable topological space
      has a countable subcover
    \end{fact}
    \begin{proof}
      \sketch{\attn todo: check.
      (A.4 in this text, or cross-reference)}
    \end{proof}
    \[
    \exists \{p_j\}_1^\infty \subset M \st
    \{U_j:=U_{p_j}\}_1^\infty \text{ covers } M.
    \]
    Also put $\varphi_j := \varphi_{p_j}$.
    By remark~\ref{rem:01:open-is-submani},
    $U_j$ is a topological $n$-manifold $\forall j\in\mathbb N$,
    so by special case,
    $\exists \mathcal E_j$,
    a countable basis of precompact coordinate balls for
    $\topo_{U_j}$, $\forall j\in\mathbb N$,
    and $\cup_{j\in\mathbb N} \mathcal E_j$ is countable
    as a countable union of countable sets.
    Moreover and also,
    $\mathcal E:= \cup_j\mathcal E_j$ a basis for $\topo_M$.
    \sketch{$\attn$ ??? this is not obvious to me}

    \sketch{
      unless... \\
      $\mathscr B$ a basis for $(X,\topo)$ iff
      $\forall U\in\topo\exists S\subset \mathscr B$
      s.t. $\cup S=U$
      $\attn$ check/cross-reference \\
      since the latter is equivalent to
      \[
      \forall U\in\topo\forall p\in U\exists V\in\mathscr B\st
      p\in V\land V\subset U,
      \]
      witch, in turn, can be satisfied, writing
      $\mathcal E:= \cup_j\mathcal E_j$
      by observing
      \[
      \forall U\in\topo_M\forall p\in U \exists j\in\mathbb N\st
      p \in U_j.
      \]
      now $\mathcal E_j$ a basis for $U_j$,
      $U\cap U_j$ open in $U_j$, and $p\in U\cap U_j$
      together imply
      \[
      \exists V\in\mathcal E_j\subset \mathcal E \st
      p\in V\land V\subset U\cap U_j.
      \]
    }
    \sketch{\loopy open sets in subspace topology of an open set
      are open sets in the original topology\loopy}
    \sketch{\attn still loopy on this one. revisit.}
    For $V\in\mathcal E$,
    $\overline{V}^{U_j}$, the closure of $V$ in $U_j$
    for some (i.e. the corresponding) $j\in\mathbb N$
    (i.e. $\st V\in\mathcal E_j$),
    is homeomorphic to a closed, bounded subset of $R^n$,
    hence compact, so $\overline{V}^{U_j}$ closed in $M$.
    \sketch{??? because compact subsets of a Haus. spc. are closed?
      double-check/cross-refrence.}
    Conclude
    \begin{enumerate}[label=(\roman*)]
    \item\label{lem:01:tm-basis:pf:conclude-01}
      $\overline{V}^{U_j} = \overline{V}^M$
      \sketch{needs check ... the above gives $\supset$,
        but the reverse isn't here verbatim}
    \item
      $\overline{V}^M$ compact
      \sketch{follows from
        \ref{lem:01:tm-basis:pf:conclude-01}
        since compact in subspace implies compact in space
      }
      (i.e. --- elements of $\mathcal E$ precompact in
      $\topo_M$ as well).
    \end{enumerate}
  \end{itemize}
\end{proof}

\begin{defn}
  a topological space, $X$, is \emph{locally compact} if
  \[
  \forall p\in X\exists U\nbhd p
  \exists\text{ compact } K\subset M\st U\subset K.
  \]
\end{defn}

\begin{rem}{D}\label{rm:01:haus-lcpt}
  if topological space, $X$, Hausdorff,
  then $X$ locally compact iff $X$ has a basis of
  precompact open sets
\end{rem}
\begin{proof}
  \sketch{\attn todo.
    text contains a reference to another book by lee.
    perhaps this can be checked directly?
  }
\end{proof}

\begin{cor}
  every topological manifold is locally compact
\end{cor}
\begin{proof}
  \ref{lem:01:tm-basis} and remark~\ref{rm:01:haus-lcpt}
\end{proof}

\section{Connectivity}
\begin{defn}
  a topological space, $X$, is
  \begin{itemize}
  \item
    \emph{connected} if
    \[
    \nexists U,V\in\topo_X \st
    U\neq\emptyset \land V\neq\emptyset\land
    U\cup V=X\land U\cap V=\emptyset
    \]
  \item
    \emph{path connected} if
    \[
    \forall p,q\in X
    \exists \gamma: [0,1]\rightarrow X \st
    \gamma(0) = p \land \gamma(1) = q \land
    \gamma \text{ continuous}
    \]
  \item
    \emph{locally path connected} if
    $X$ has a basis for path connected (open) sets
  \end{itemize}
\end{defn}

\begin{prop}\label{prop:1.8}\label{prop:01:conn-mani}
  for topological manifold, $M$,
  \begin{enumerate}[label=(\alph*)]
  \item\label{prop:01:conn-mani:lpconn}
    $M$ locally path connected
  \item\label{prop:01:conn-mani:conn-pconn-equiv}
    $M$ connected iff $M$ path connected
  \item\label{prop:01:conn-mani:conn-pconn-comp-equiv}
    the connected components of $M$ are its path-connected components
  \item\label{prop:01:conn-mani:comp}
    \begin{enumerate}[label=(\roman*)]
    \item\label{prop:01:conn-mani:comp:count}
      $M$ has at most countably many components
    \item\label{prop:01:conn-mani:comp:open}
      each component of $M$ is open in $M$, and
    \item\label{prop:01:conn-mani:comp:tm}
      each component is a connected topological manifold
    \end{enumerate}
  \end{enumerate}
\end{prop}
\begin{proof}\
  \begin{itemize}
  \item[\ref{prop:01:conn-mani:lpconn}]
    follows from \ref{lem:01:tm-basis},
    since balls in euclidean space are path connected
  \item[\ref{prop:01:conn-mani:conn-pconn-equiv},
    \ref{prop:01:conn-mani:conn-pconn-comp-equiv},
    \ref{prop:01:conn-mani:comp}\ref{prop:01:conn-mani:comp:open}]
    Recall the following
    \begin{fact}
      for a locally path connected topological space, $X$,
      \begin{enumerate}[label=(\alph*)]
      \item
        the connected components of $X$
        (i.e. the equivalence classes of $\sim$, where
        $p\sim q$ if $\exists S\subset X$ such that
        $S$ connected and $p, q\in S$)
        are open in $X$
      \item
        the path connected components of $X$ are the same
        as the connected components of $X$.
      \item
        $X$ connected iff $X$ path connected
      \end{enumerate}
    \end{fact}
    \sketch{$\attn$ the above is available in this text as A.16.
      include proof or cross-reference.\\
      also handle loose ends of applying this and
      \ref{prop:01:conn-mani:lpconn}
    }
  \item[\ref{prop:01:conn-mani:comp}\ref{prop:01:conn-mani:comp:count}]
    \ref{prop:01:conn-mani:comp}\ref{prop:01:conn-mani:comp:open}
    allows us to view the components of $M$ as and open cover, $C$,
    and $M$ second countable implies
    the cover has a countable subcover $S\subset C$.
    \sketch{$\attn$ double-check/cross-reference that every
      open cover of a second-countable space has a countable
      subcover
    }
    since $U\cap V=\emptyset, \forall U,V\in C$
    by definition of connected components as equivalence classes,
    $\cup S = M = \cup C \Rightarrow S=C$
    \sketch{$\attn$ todo: spell out contradiction}
  \end{itemize}
\end{proof}

\section{Fundamental Groups of Manifolds}
\begin{prop}
  the fundamental group, $\pi_1(M)$,
  of any topological manifold, $M$, is countable
\end{prop}
\begin{proof}
  by \ref{lem:01:tm-basis}, choose
  $\mathscr B\subset\mathscr P(M)$,
  a cover of $M$ by coordinate balls.
  for any $B, B'\in\mathscr B$,
  $B\cap B'$ open hence a topological manifold
  by remark~\ref{rem:01:open-is-submani}
  and has at most countably many components by
  \ref{prop:01:conn-mani}\ref{prop:01:conn-mani:comp}
  \ref{prop:01:conn-mani:comp:count},
  and each component of $B\cap B'$ is path connected by
  \begin{fact}
    for a locally path connected topological space, $X$,
    the path connected components of $X$ are the same
    as the connected components of $X$.
  \end{fact}
  \sketch{$\attn$ the above is available in this text as A.16.
    include proof or cross-reference.
  }
  So get $\mathcal X$ by choice:
  choose one point from each component
  of $B\cap B'$ for all $(B, B')\in\mathcal B^2$,
  and let the union of these points be $\mathcal X$
  such that $\mathcal X$ is countable,
  then for each $B\in\mathcal B$ and $x, x'\in B$,
  let $p_{x,x'}^B$ be a path from $x$ to $x'$
  (i.e. a continuous map $[0, 1]\rightarrow B$).
  Choose $q\in\mathcal X$ as a base point for loops in $\pi_1(M)$
  --- i.e. consider $\pi_1(M, q)$ ---
  and let a \emph{special loop} be one that's a finite product
  (considering the fundamental group as multiplicative)
  of paths of the the form $p_{x,x'}^B$.
  Since the set of finite subsets of a countable set is countable,
  \sketch{$\attn$ make sure}
  the set of special loops is countable.
  The rest of this proof consists of showing every element of
  $\pi_1(M, q)$ homotopic to a special loop.
  \paragraph{}
  Fix any path $f:[0,1]\rightarrow M$ with $f(0)=f(1)=q$
  (i.e. any loop at $q$).
  Now $\{f^{-1}(B) \mid B\in\mathcal B\}$ an open cover
  of compact $[0, 1]$, so
  \sketch{$\attn$
    take care about constructing the following $\attn$\\
    --- possibly use some facts about preimg.s of conn. sets}
  \[
  \exists 0=a_0<a_1<\cdots<a_n=1\st
  \forall j\in[n]\exists B\in\mathcal B
  \big[[a_{j-1}, a_j]\subset f^{-1}(B)\big].
  \]
  \sketch{$\attn$Finish$\attn$
    get some paths $\{f_j\}_1^n$ and construct homotopy
    to special paths using\\
    1. path-conn. of each $B_{j-1}\cap B_j$, $B_j\cap B_{j+1}$ \\
    2. homotopy between any two paths in $B_j$ by homeomorphism
    to Euclidean ball
  }
\end{proof}

\section{Smooth Structures}
\begin{defn}
  for open $U\subset R^n, V\subset R^m$,
  $F: U\rightarrow V$ is \emph{smooth}
  (or $C^\infty$ or \emph{infinitely differentiable})
  if it has continuous partial derivatives of all orders
  \sketch{$\attn$ todo ---
    calculus cross-reference:
    at a minimum, provide definition of
    ``partial derivatives of all orders''
  }
  and if, additionally, $F$ a bijection with smooth inverse,
  then $F$ is a \emph{diffeomorphism}
\end{defn}

\begin{defn}
  for topological manifold, $M^n$, and
  charts, $(U,\varphi), (V,\psi) \st U\cap V \neq \emptyset$,
  $\psi\circ\varphi^{-1}:\varphi(U\cap V)\rightarrow \psi(U\cap V)$
  is the \emph{transition map} from $\varphi$ to $\psi$.
  \sketch{$\loopy$ notice directionality of chart
    ($M\rightarrow U$ \underline{not} $U\rightarrow M$)
    is different from some other treatments
    }
\end{defn}

\begin{rem}{E}
  the transition map is a homemorphism
  as the composition of homeomorphisms
\end{rem}

\begin{defn}\
  \begin{itemize}
  \item
    two charts, $(U,\varphi)$, $(V,\psi)$ are
    \emph{smoothy compatible} if
    \[
    U\cap V=\emptyset \lor
    \psi\circ\varphi^{-1}\text{ a diffeomorphism}
    \]
  \item
    an \emph{atlas} for topological manifold $M^n$ is a set of charts
    \[
    \{(U_\alpha,\varphi_\alpha)\}_{\alpha\in A}
    \st \cup_{\alpha\in A} U_\alpha = M
    \]
  \item
    an atlas, $\mathscr A$, is a \emph{smooth atlas}
    if $C_1, C_2$ smoothly compatible $\forall C_1, C_2\in\mathcal A$.
  \end{itemize}
\end{defn}

\begin{rem}{F}\label{rem:01:sm-atlas-equiv}
  formally weaker
  \[
  \psi\circ\varphi^{-1}\text{ smooth},
  \forall (U,\varphi),(V,\psi)\in\mathcal A
  \]
  suffices to show $\mathcal A$ smooth
\end{rem}
\begin{proof}
  $(\psi\circ\varphi)^{-1}=\varphi\circ\psi^{-1}$
\end{proof}

\sketch{$\loopy$
  just as connected components can be defined as maximal
  (under inclusion), connected sets of a topological space
  \underline{or} as equivalence classes as in appendix of this text,
  smooth atlases also may be defined either by maximality or
  with an equivalence relation...\\
  ... here, we'll use maximality
}

\begin{defn}\
  \begin{itemize}
  \item
    a smooth atlas is \emph{maximal} (or \emph{complete})
    if it is not properly (set) included in another atlas
    (i.e. $\nexists$ a smooth atlas $\mathscr A'$
    such that $\mathscr A\subsetneq \mathscr A'$).
  \item
    a \emph{smooth structure} on a topological manifold
    is a maximal smooth atlas (called a \emph{differentiable structure}
    or \emph{$C^\infty$ structure} by some)
  \item
    a \emph{smooth manifold} is a pair $(M, \mathscr A)$,
    where $M$ a topological manifold and $\mathscr A$
    a smooth structure on $M$
    ($\mathscr A$ omitted when clear from context), and
    \emph{smooth manifold structure} will be used to describe,
    connotatively, a manifold topology together with a smooth structure
    --- the additional information that separates an object in the
    class of manifolds from that in the class smooth manifolds
  \end{itemize}
\end{defn}

\begin{rem}{}
  there are topological manifolds without smooth structure
  (although the first such example wasn't found until
  1960 by Kervaire).
\end{rem}

\begin{lem}\label{lem:1.10}\label{lem:01:max-atlas}
  for any topological manifold, $M^n$,
  \begin{enumerate}[label=(\alph*)]
  \item\label{lem:01:max-atlas:exists}
    $\forall$ smooth atlas $\mathscr A$ on $M$,
    $\exists!$ maximal smooth atlas $\overline{\mathscr A}$
    on $M$ $\st \mathscr A\subset\overline{\mathscr A}$
  \item\label{lem:01:max-atlas:pair-uniq}
    [continuing with notation from \ref{lem:01:max-atlas:exists}]
    $\forall$ smooth atlas $\mathscr A_1,\mathscr A_2$ on $M$,
    $\overline{\mathscr A_1} = \overline{\mathscr A_2}$
    $iff$ $\mathscr A_1 \cap \mathscr A_2$ a smooth atlas
  \end{enumerate}
\end{lem}
\begin{proof}\
  \begin{itemize}
  \item[\ref{lem:01:max-atlas:exists}]
    Put
    \[
    \overline{\mathscr A}:=
    \{\text{chart } (U,\varphi)\text{ on } M\mid
    (U,\varphi)\text{ sm. compat. }
    (V,\psi),\forall(V,\psi)\in\mathscr A\}.
    \]
    Fix
    \[
    (U,\varphi), (V,\psi)\in \overline{\mathscr A}
    \st U\cap V\neq \emptyset.
    \]
    Choose $p\in U\cap V\subset M$, and let
    $x=\varphi(p)\in\varphi(U\cap V)\subset R^n$.
    Since $\cup_{\alpha\in A} W_\alpha = M$,
    where $\mathscr A=\{(W_\alpha,\theta_\alpha)\}_{\alpha\in A}$,
    choose $(W,\theta)\in\mathscr A\st p\in W$.
    Now by definition of $\overline{\mathscr A}$,
    $\theta\circ\varphi^{-1}$ a diffeomorphism --- i.e.
    $\theta\circ\varphi^{-1}: \varphi(U\cap W)\rightarrow\theta(U\cap W)$
    is smooth with smooth inverse $\varphi\circ\theta^{-1}$,
    and similarly,
    $\theta\circ\psi^{-1}$ a diffeomorphism --- i.e.
    $\theta\circ\psi^{-1}: \psi(V\cap W)\rightarrow\theta(V\cap W)$
    is smooth with smooth inverse $\psi\circ\theta^{-1}$.
    In particular, $\theta\circ\varphi^{-1}$ and $\theta\circ\psi^{-1}$
    are smooth, and because $p\in U\cap V\cap W\neq\emptyset$,
    $\psi\circ\varphi^{-1}$ is smooth on
    $\varphi(U\cap V\cap W)\nbhd x$,
    \sketch{$\loopy$ careful about neighborhood domains/ranges}
    where it may be defined as
    the composition of smooth maps
    $\psi\circ\varphi^{-1} = (\psi\circ\theta^{-1})\circ(\theta\circ\varphi^{-1})$.
    Because $p\in U\cap V$ arbitrary, the preceeding shows
    $\psi\circ\varphi^{-1}$ smooth on all of $U\cap V$
    \sketch{$\loopy$
      b/c smoothness a ``local'' property, a property s.t. if true
      on nbhd. of every pt., true for entire manifold.\\
      contrast with point-wise properties, and why not say
      ``smooth at every point''?
      }
    so $\overline{\mathscr A}$ a smooth atlas by
    remark~\ref{rem:01:sm-atlas-equiv}.
    \paragraph{}
    $\overline{\mathscr A}$ maximal by definition, for if
    $\widetilde{\mathscr A}\subset\overline{\mathscr A}$
    a smooth atlas on $M$, and
    $\exists C\in \widetilde{\mathscr A}\setminus\overline{\mathscr A}$,
    then $C$ smoothly compatible $C'$,
    $\forall C'\in\overline{\mathscr A}\subset\widetilde{\mathscr A}$
    by definition of smooth atlas $\Rightarrow$
    $C\in\overline{\mathscr A}$ by definition of
    $\overline{\mathscr A}$,
    $\lightning$ choice of $C$.
    \paragraph{}
    For uniqueness, choose any maxiaml, smooth atlas
    $\mathscr B\supset\mathscr A$.
    Then $C$ smoothly compatible $C'$,
    $\forall C'\in\mathscr A\subset\mathscr B$
    by definition of smooth atlas ($\mathscr B$),
    so $\mathscr B\subset\overline{\mathscr A}$
    and maximality of $\mathscr B$ implies
    $\mathscr B=\overline{\mathscr A}$.
  \item[\ref{lem:01:max-atlas:pair-uniq}]
    \sketch{$\attn$todo$\attn$}
    \sketch{($\Leftarrow$):
      s'th like
      \[
      \overline{\mathscr A_1} =
      \overline{\mathscr A_1 \cup \mathscr A_2} =
      \overline{\mathscr A_2}
      \]
      by observing $\lightning$s. \\
      ($\Rightarrow$): Put
      \[
      \overline{\mathscr A} := \overline{\mathscr A_1} =
      \overline{\mathscr A_2}
      \]
      show $\mathscr A_1\cup\mathscr A_2$ $\lnot$ sm.
      $\Rightarrow$ $\overline{\mathscr A}$ $\lnot$ sm,
      for $\mathscr A_1\cup\mathscr A_2 \subset \overline{\mathscr A}$.
    }
  \end{itemize}
\end{proof}

\section{Local Coordinate Representations}
\begin{defn}\
  \begin{itemize}
  \item
    for smooth manifold $(M, \mathscr A)$, any
    $(U,\varphi)\in\mathscr A$ is a \emph{smooth chart},
    $\varphi$ is a \emph{smooth coordinate map}, and
    $U$ is a \emph{smooth coordinate domain}
    (or \emph{smooth coordinate neighborhood}).
  \item
    a \emph{smooth coordinate ball}, $U$, is
    the smooth coordinate domain of a smooth chart,
    $(U, \varphi)$ such that $\varphi(U)\in R^n$ a ball.
  \end{itemize}
\end{defn}

\begin{lem}\label{lem:1.11}
  every smooth manifold has a countable basis
  of precompact, smooth coordinate balls
\end{lem}
\begin{proof}
  \sketch{$\attn$ todo --- adapt \ref{lem:01:tm-basis}
    (Ex. 1.5)
  }
\end{proof}

\begin{defn}
  when chart $(U,\varphi)$ on $M^n$ understood from context,
  for $p\in U$, ``$(x^1,\ldots,x^n)$ is the (local) coordinate
  representation for $p$'' or ``$p=(x^1,\ldots,x^n)$ in
  local coordintes'' means
  $\varphi(p)=(x^1,\ldots,x^n)\in R^n$.
\end{defn}

In other words, for a chart,
$(U,\varphi: U\rightarrow\widetilde{U})$,
$\varphi$ is considered to be the identity map and removed from
notation while existing in local context.
\sketch{$\loopy$ attempts to formalize the (common) convention?
  recall $\varphi$ bij. $U\leftrightarrow\widetilde{U}$
}
Take care, however, not to confuse general statements about
open $U\subset M$ with those that are specific to a particular
coordinate map $\varphi:U\rightarrow\widetilde{U}\subset R^n$.

\sketch{omitting some examples and § markers}

\begin{nota}
  (Einstein summation convention) ---
  if an index appears twice in a monomial term,
  once as a superscript and once as subscript,
  then that term denotes a summation
  (e.g. $x^iE_j$ denotes $\sum_{i\in I} x^i E_i$).
\end{nota}

\section{Manifolds with Boundary}

\begin{defn}\
  \begin{itemize}
  \item
    \emph{an $n$-dimensional topological manifold with boundary}
    is a second-countable, Hausdorff topological space
    such that every point has a neighborhood homemorphic to
    an open subset of $\mathbb H^n$, where
    \[\mathbb H^n = \{(x^1,\ldots,x^n)\in R^n\mid x^n\geq 0\}\]
    with subspace topology induced by $R^n$ is
    the \emph{upper half-space}
    (i.e. --- a topological manifold with boundary is just as
    a topological manifold, except with charts into $\mathbb H^n$
    instead of $R^n$)
  \item
    a chart $(U,\varphi)$ on a topological manifold with boundary
    is an \emph{interior chart} if
    $\varphi(U)\subset\Int\mathbb H^n$
    and a \emph{boundary chart} otherwise
    (i.e. if $\varphi(U)\cap\mathbb \bdry H^n\neq\emptyset$),
    where
    \[
    \Int\mathbb H^n = \{(x^1,\ldots,x^n)\in R^n\mid x^n > 0\}
    \text{ and }
    \bdry\mathbb H^n = \{(x^1,\ldots,x^n)\in R^n\mid x^n = 0\}
    \]
  \item
    for any $A\subset R^n$, $f: A\rightarrow R^m$ smooth if
    \[
    \forall p\in A\exists U\nbhd p \underline{\text{ in } R^n},
    g: U\rightarrow R^m \st
    g \text{ is }C^\infty\land \restr{g}{A} = \restr{f}{U}
    \]
    (i.e. --- smooth maps on Euclidean subsets must be
    smooth at boundary), so for $U\in\topo_{\mathbb H^n}$,
    $F: U\rightarrow R^m$ is smooth if
    \[
    \forall x\in U \exists\text{open } V\subset R^n
    \exists\text{smooth }\widetilde{F}: V\rightarrow R^m
    \st F=\widetilde{F} \text{ on } V\cap U
    \]
  \end{itemize}
\end{defn}

\begin{rem}
  for smooth $F:U\subset\mathbb H^n\rightarrow R^m$,
  partial derivates of (extensions of) $F$ on
  $U\cap\bdry\mathbb H^n$ are independent of the choice of extension
\end{rem}
\begin{proof}
  ($\attn$ sketch) $\restr{F}{U\cap\Int\mathbb H^n}$ smooth
  in the sense of partial derivates of all orders.
  moreover, differentiability implies continuity,
  so unique values of partials on $U\cap\bdry\mathbb H^n$
  are provided by taking limits of partials evaluated at
  points in $U\cap\Int H^n$.
\end{proof}

\sketch{skipping some --- In particular,
  not until Chap. 7 will it be proved that \\
  $\Int M$ points sent to $\Int\mathbb H^n$ by charts and \\
  $\bdry M$ points sent to $\bdry\mathbb H^n$ by charts \\
  are disjoint
}
