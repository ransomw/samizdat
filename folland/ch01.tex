\chapter{Measures}

\section{Intro}
can't define ``measures'' (text gives hypothetical definition)
on \emph{all} subsets of $R^n$, not even with finite additivity,
due to, e.g., Banach-Tarski.

\section{$\sigma$--Algebras}\label{sec:01:sigma-algebras}
\begin{defn}\
  \begin{itemize}
  \item an \emph{algebra of sets} on a set ${X\neq\emptyset}$
    is ${\mathcal{A}\subset\mathscr{P}(X)}$ such that
    ${\mathcal{A}\neq\emptyset}$ and $\mathcal{A}$
    closed under finite unions and compliments
  \item a \emph{$\sigma$--algebra} is an algebra of sets
    closed under countable unions
  \end{itemize}
\end{defn}

\begin{rem}{A}
  ${\cap E=\setcomp{\left(\cup \setcomp{E}\right)}}$
  implies algebras closed under intersections as well
\end{rem}

\begin{rem}{B}
  $\emptyset, X$ included in any algebra on $X$ since
  \[
  \mathcal{A}\neq\emptyset \Rightarrow \exists E\in\mathcal{A}
  \Rightarrow E\cup\setcomp{E}=X\in\mathcal{A}
  \Rightarrow \emptyset=\setcomp{X}\in\mathcal{A}
  \]
\end{rem}

\begin{rem}{C}\label{rem:01:sigma-algebra-equiv}
  formally weaker ``closed under countable disjoint unions''
  is logically equivalent to statement of definition of
  $\sigma$--algebra, for given algebra $\mathcal{A}$ and
  ${\{E_i\}\subset\mathcal{A}}$, put
  \[
  F_i=E_i\setminus\left(\cup_{j=1}^{i-1} E_j\right)=
  E_i\cap\setcomp{\left(\cup_{j=1}^{i-1} E_j\right)}
  \]
  such that
  ${\{F_i\}\subset\mathcal{A}}$ and
  ${\cup E_i = \cup F_i}$.
  \sketch{\attn iron out logical equivalence.
    particularly, ${\{F_i\}\subset\mathcal{A}}$
    from revised hypothesis}
\end{rem}

\begin{defn}
  for uncountable $X$, the
  \emph{$\sigma$-algebra of countable or co-countable sets}
  is
  \[
  \mathcal{A} =
  \{ E\subset X\mid E\text{ countable }\lor \setcomp{E}\text{ countable}\}
  \]
\end{defn}

\begin{rem}{}
  the intersection of any set of $\sigma$-algebras is a $\sigma$-algebra
  so there is a unique smallest $\sigma$-algebra containing any
  subset of ${\mathscr{P}(X)}$
\end{rem}

\begin{defn}
  for ${\mathcal{E}\subset\mathscr{P}(X)}$,
  the $\sigma$-algebra \emph{generated} by $\mathcal{E}$ is
  \[
  \mathcal{M}(\mathcal{E}) =
  \cap\{\mathcal{A}\subset\mathscr{P}(X)\mid
  \mathcal{A}\text{ a $\sigma$-algebra} \land
  \mathcal{E}\subset\mathcal{A}\}
  \]
\end{defn}

\begin{lem}\label{lem:1.1}\label{lem:01:gensa}
  \[
  \mathcal{E}\subset\mathcal{M}(\mathcal{F})
  \Rightarrow
  \mathcal{M}(\mathcal{E)}\subset\mathcal{M}(\mathcal{F})
  \]
\end{lem}

\begin{defn}
  for topological space, $X$, the
  \emph{Borel $\sigma$-algebra} on $X$, $\mathcal{B}_X$,
  is the $\sigma$-algebra generated by the open sets in $X$
\end{defn}

\begin{rem}{}
  an equivalent definition of the Borel $\sigma$-algebra
  is that it's the $\sigma$-algebra generated by the
  closed sets in $X$
\end{rem}

\begin{defn}
  \emph{Borel sets} are elements of $\mathcal{B}_X$.
  \begin{itemize}
  \item a countable $\cap$ of open sets is called a
    $G_\delta$ set,
    a countable $\cup$ of closed sets is an $F_\sigma$ set
  \item a countable $\cup$ of $G_\delta$ sets is called a
    $G_{\delta\sigma}$ set,
    a countable $\cap$ of $F_\sigma$ sets is an $F_{\sigma\delta}$ set,
    and so on
  \end{itemize}
\end{defn}

\begin{prop}\label{prop:1.2}\label{prop:01:borelrealsgen}
  $\mathcal{B}_{\mathbb{R}}$ is generated by
  \begin{itemize}
  \item[(a)] closed intervals ${\{[a, b]\}}$
  \item[(b)] open intervals ${\{(a, b)\}}$
  \item[(c)] half-open intervals
    ${\{\halfopen{a}{b}\}\cup\{\openhalf{a}{b}\}}$
  \item[(d)] open \emph{rays}
    ${\{(a,\infty)\}\cup\{(-\infty,a)\}}$
  \item[(e)] closed rays
    ${\{\halfopen{a}{\infty}\}\cup\{\openhalf{-\infty}{b}\}}$
  \end{itemize}
\end{prop}
\begin{proof}
\sketch{todo: Ex. 2}
\end{proof}

\begin{defn}
  the \emph{product $\sigma$-algebra},
  ${\bigotimes_{\alpha\in A} \mathcal{M}_\alpha}$
  of an indexed set of $\sigma$-algebras ${\mathcal{M}_\alpha}$,
  each on a set $X_\alpha$ is the $\sigma$-algebra generated by
  \[
  \{\pi_\alpha^{-1}(E_\alpha)\mid E_\alpha\in\mathcal{M}_\alpha, \alpha\in A\}
  \]
  where ${\pi_\alpha: \prod_{\alpha\in A} X_\alpha\rightarrow X_\alpha}$
  are the usual coordinate maps.
\end{defn}

\begin{prop}\label{prop:1.3}\label{prop:01:prodsachar}
  if $A$ countable, ${\bigotimes_{\alpha\in A} \mathcal{M}_\alpha}$
  is the $\sigma$-algebra generated by
  ${\left\{\prod_{\alpha\in A} E_\alpha\mid E_\alpha\in\mathcal{M}_\alpha\right\}}$.
\end{prop}
\begin{proof}
  put
  \[
  S=
  \{\pi_\alpha^{-1}(E_\alpha)\mid E_\alpha\in\mathcal{M}_\alpha, \alpha\in A\}
  \text{ and }
  T=
  \left\{\prod_{\alpha\in A} E_\alpha\mid E_\alpha\in\mathcal{M}_\alpha \right\}
  \]

  By definition of coordinate maps,
  ${\pi_\alpha^{-1}(E_\alpha)=\prod_{\beta\in A} E_\beta}$,
  where ${E_\beta=X_\beta \text{ for }\beta\neq\alpha}$,
  so ${S\subset T \Rightarrow \mathcal{M}(S)\subset\mathcal{M}(T)}$.

  Observe
  \[
  \prod_{\alpha\in A} E_\alpha = \bigcap_{\alpha\in A} \pi_\alpha^{-1}(E_\alpha)
  \]
  for indeed
  \[
  f\in \prod_{\alpha\in A} E_\alpha \iff
  f(\alpha)\in E_\alpha, \forall \alpha\in A \iff
  f\in\pi_\alpha^{-1}(E_\alpha), \forall \alpha\in A.
  \]
  So since $A$ countable by hypothesis and since
  $\sigma$-algebras are closed under countable intersections,
  ${T\subset\mathcal{M}(S)}$, and \ref{lem:01:gensa}
  implies ${\mathcal{M}(T)\subset\mathcal{M}(S)}$.
\end{proof}

\begin{prop}\label{prop:1.4}
  \[
  \mathcal{M}_\alpha=\mathcal{M}(\mathcal{E}_\alpha),\forall\alpha\in A
  \Rightarrow
  \bigotimes_{\alpha\in A}\mathcal{M}_\alpha=\mathcal{M}(\mathcal{F}_1)
  \]
  where
  \[
  \mathcal{F}_1=
  \{\pi_\alpha^{-1}(E_\alpha)\mid E_\alpha\in\mathcal{E}_\alpha , \alpha\in A\}
  \]
  And
  \[
  A \text{ countable}\land X_\alpha\in E_\alpha, \forall\alpha\in A
  \Rightarrow
  \bigotimes_{\alpha\in A}\mathcal{M}_\alpha=\mathcal{M}(\mathcal{F}_2)
  \]
  where
  \[
  \mathcal{F}_2=
  \left\{\prod_{\alpha\in A} E_\alpha\mid E_\alpha\in\mathcal{E}_\alpha\right\}.
  \]
\end{prop}

\begin{prop}\label{prop:1.5}
  for metric spaces ${X_1,\ldots ,X_n}$ and ${X=\prod_1^n X_i}$
  with product metric,
  \[
  \bigotimes_1^n \mathcal{B}_{X_i} \subset \mathcal{B}_X
  \land X_i \text{ seperable } \forall i\in [n]
  \Longrightarrow
  \mathcal{B}_X \subset \bigotimes_1^n \mathcal{B}_{X_i}
  \]
  and hence equality
  ${\mathcal{B}_X = \bigotimes_1^n \mathcal{B}_{X_i}}$.
\end{prop}
\begin{proof}
  choose countable, dense ${C_i\subset X_i}$ for each ${i\in [n]}$,
  and put
  \[ \mathcal{E}_i=\{B(r, p)\mid r\in \mathbb{Q}, p\in C_i\}. \]
  Then every open set in $X_i$ is the (countable) union
  of elements of $\mathtt{\mathcal{E}_i}$
  (since $\mathcal{E}_i$ countable).
  \sketch{double-check this ---
    probably uses triangle inequality and union of set of elements
    of $\mathcal{E}_i$, one for each element of the open set}

  \sketch{%
    text further states without proof that
    \\
    points in $\mathtt{X}$ with $\mathtt{j}^{\text{th}}$ coordinate
    in $\mathtt{C_j}$ for all $\mathtt{j\in [n]}$ is a countable
    dense subset of $\mathtt{X}$
  }

  \sketch{after these intermediate details are sorted out...}

  Conclude ${\mathcal{M}(\mathcal{E}_j) = \mathcal{B}_{X_j}}$ and
  ${\mathcal{B}_X = \left\{\prod_1^n E_j\mid E_j\in\mathcal{E}_j\right\}}$
  and apply \ref{prop:1.4}

  \sketch{spell out the application better}
\end{proof}

\begin{cor}\label{cor:1.6}\label{cor:01:borelreals}
  \[ \mathcal{B}_{R^n} = \bigotimes_1^n \mathcal{B}_R \]
\end{cor}

\begin{defn}
  an \emph{elementary family} ${\mathcal{E}\subset \mathscr{P}(X)}$
  is a collection of subsets such that
  \begin{itemize}
  \item[(1)] ${\emptyset\in\mathcal{E}}$
  \item[(2)] ${E, F\in\mathcal{E}} \Rightarrow E\cap F\in \mathcal{E}$
  \item[(3)]
    ${E\in\mathcal{E}\Rightarrow \exists F_1,\ldots,F_n\in\mathcal{E} \st \setcomp{E} = \sqcup_i F_i}$
  \end{itemize}
\end{defn}

\begin{prop}\label{prop:1.7}\label{prop:01:elemfamalg}
  the set of finite disjoint unions of members of an elementary family
  is an algebra
\end{prop}
\begin{proof}
  Fix elementary family $\mathcal{E}$, choose ${A, B\in \mathcal{E}}$,
  ${\{C_j\}_1^J\subset\mathcal{E}\st \setcomp{B}=\sqcup_1^J C_j}$.
  Then ${A\setminus B=\sqcup_1^J(A\cap C_j)}$
  and ${A\cup B=(A\setminus B)\sqcup B}$
  are disjoin unions of elements of $\mathcal{E}$.
  Inductively, ${\cup_1^n A_j}$ is a disjoint union of elements of
  $\mathcal{E}$ for all finite ${\{A\}_1^n\subset\mathcal{E}}$.

  Now fix disjoint ${\{A_i\}_{i=1}^n\subset\mathcal{E}}$
  and choose ${\{B_i^j\}_{j=1}^{J_i}}$ such that
  ${\setcomp{A_i}=\sqcup_{j=1}^{J_i} B_i^j,\forall i\in [n]}$.
  Then,
  \[
  \setcomp{\left(\bigsqcup_{i=1}^n A_i\right)} =
  \bigcap_{i=1}^n\bigcup_{j=1}^{J_i} B_i^j \stackrel{*}{=}
  \bigcup\big\{B_1^{j_1}\cap\cdots\cap B_n^{j_n}\mid j_i\in [J_i], i \in [n]\big\},
  \]
  \begin{equation}
    x \text{ in the set} \iff
    \forall i\exists j\st x\in B_i^j
    \tag{*}
  \end{equation}
  is the union of elements of $\mathcal{E}$, hence the union of
  disjoint elements of $\mathcal{E}$ by the previous part of this proof.
\end{proof}

\section{Measures}
\begin{defn}
  a \emph{measure} is a $\sigma$-algebra
  ${\mathcal{M}\subset\mathscr{P}(X)}$
  (or on ${(X, \mathcal{M})}$ or on $X$ if $\mathcal{M}$ understood)
  is a function ${\mu:\mathcal{M}\rightarrow[0,\infty]}$
  such that
  \begin{enumerate}[label=(\roman*)]
  \item ${\mu(\emptyset)=0}$
  \item\label{defn:measure:itm:2}
    ${\{E_j\}_1^\infty\subset\mathcal{M}\text{ disjoint} \Rightarrow \mu(\cup_j E_j) = \sum_j\mu(E_j)}$
  \end{enumerate}
  where \ref{defn:measure:itm:2} above is called
  \emph{countable additivity}.
\end{defn}

\begin{rem}{}
  countable additivity implies \emph{finite additivity} --- i.e.
  \[
  E_1,\ldots,E_n\text{ disjoint}\Rightarrow
  \mu(\cup_1^n E_j) = \sum_1^n\mu(E_j)
  \]
\end{rem}

\begin{defn}\
  \begin{itemize}
  \item given a set, $X$, and a $\sigma$-algebra,
    ${\mathcal{M}\subset\mathscr{P}(X)}$,
    ${(X,\mathcal{M})}$ is called a
    \emph{measurable space}, and the sets in $\mathcal{M}$
    are called \emph{measurable sets}.
  \item given a measure $\mu$ on a measurable space
    ${(X,\mathcal{M})}$,
    ${(X,\mathcal{M}, \mu)}$ is called a
    \emph{measure space}
  \item $\mu$ \emph{finite} iff ${\mu(X)<\infty}$
  \end{itemize}
\end{defn}

\begin{rem}{}
  \[ \mu\text{ finite}\Rightarrow\mu(E)<\infty,\forall E\in\mathcal{M} \]
\end{rem}

\begin{defn}\
  \begin{itemize}
  \item $\mu$ \emph{$\sigma$-finite} if
    \[
    \exists \{E_j\}_1^\infty\subset\mathcal{M}\st
    X=\cup_1^\infty E_j \land \mu(E_j)<\infty,\forall j\in N
    \]
  \item ${E\subset X}$ \emph{$\sigma$-finite} for $\mu$ if
    \[
    \exists \{E_j\}_1^\infty\subset\mathcal{M}\st
    E=\cup_1^\infty E_j \land \mu(E_j)<\infty,\forall j\in N
    \]
    (sometimes said ``$E$ of $\sigma$-finite measure'')
  \item $\mu$ \emph{semifinite} if
    \[
    \forall E\in\mathcal{M}, \mu(E)=\infty
    \Longrightarrow
    \exists F\in\mathcal{M}\st
    F\subset E\land 0<\mu(F)<\infty
    \]
  \end{itemize}
\end{defn}

\begin{rem}{}
  $\sigma$-finite implies semifinite
  \sketch{Exercise 13}
  but not conversely
  \sketch{? counterexamples?}
\end{rem}

\begin{exa}\
  \begin{itemize}
  \item ${X\neq\emptyset}$, ${\mathcal{M}\subset\mathscr{P}(X)}$,
    ${f: X\rightarrow[0,\infty]}$ any function.
    Then ${\mu(E)=\sum_{x\in E} f(x)}$ is a measure,
    \[ \mu\text{ semifinite}\iff f(x)<\infty,\forall x\in X \]
    and
    \[
    \mu\text{ $\sigma$-finite}\iff
    \mu\text{ semifinite} \land
    \{x\in X\mid f(x)>0\} \text{ countable}
    \]
    (recall \ref{prop:00:fn-sums} for the second part of the
    forward direction).

    If ${f(x)=1,\forall x\in X}$, then $\mu$ is called
    \emph{counting measure},
    if
    \[
    \exists x_0\in X\st f(x_0)=1 \land
    \forall x\in X[x\neq x_0\Rightarrow f(x)=0]
    \]
    $\mu$ is called \emph{point mass} or \emph{Dirac measure} at $x_0$,
    and these two names also apply to measures on smaller
    $\sigma$-algebras ${\mathcal{M}'\subset\mathscr{P}(X)}$.
  \item for uncountable $X$, ${\mathcal{M}\subset\mathscr{P}(X)}$
    the $\sigma$-algebra of countable or co-countable sets
    \[
    \mu: \mathcal{M}\rightarrow[0,\infty]\quad
    E\mapsto\left\{
    \begin{array}{rl}
      1 & \text{if $E$ co-countable},\\
      0 & \text{if $E$ countable}
    \end{array}
    \right.
    \]
    is a measure
  \item For infinite $X$, put ${\mathcal{M}=\mathscr{P}(X)}$ and
    \[
    \mu: \mathcal{M}\rightarrow[0,\infty]\quad
    E\mapsto\left\{
    \begin{array}{rl}
      \infty & \text{if $E$ infinite},\\
      0 & \text{if $E$ finite}.
    \end{array}
    \right.
    \]
    Then $\mu$ is finitely but not countably additive.
  \end{itemize}
\end{exa}

\begin{thm}\label{thm:1.8}\label{thm:01:measure-char}
  For measure space ${(X, \mathcal{M},\mu)}$,
  \begin{enumerate}[label=(\alph*)]
  \item\label{thm:01:itm:mon}\emph{monotonicity}
    \[
    E,F\in\mathcal{M}\land E\subset F
    \Longrightarrow
    \mu(E)\leq\mu(F)
    \]
  \item\label{thm:01:itm:sub}\emph{subadditivity}
    \[
    \{E_j\}_1^\infty\subset\mathcal{M}
    \Longrightarrow
    \mu(\cup_1^\infty E_j) \leq \sum_1^\infty \mu(E_j)
    \]
  \item\label{thm:01:itm:cts-below}\emph{continunity from below}
    \[
    \{E_j\}_1^\infty\subset\mathcal{M} \land E_1\subset E_2\subset\ldots
    \Longrightarrow
    \mu(\cup_1^\infty E_j) = \lim_{j\rightarrow\infty} \mu(E_j)
    \]
  \item\label{thm:01:itm:cts-above}\emph{continunity from above}
    \[
    \{E_j\}_1^\infty\subset\mathcal{M} \land E_1\supset E_2\supset\ldots
    \land \mu(E_1)<\infty
    \Longrightarrow
    \mu(\cap_1^\infty E_j) = \lim_{j\rightarrow\infty} \mu(E_j)
    \]
  \end{enumerate}
\end{thm}
\begin{proof}\
  \begin{enumerate}[label=(\alph*)]
  \item
    \[
    \mu(F)\stackrel{\ref{defn:measure:itm:2}}{=}
    \mu(E)+\mu(F\setminus E)\geq\mu(E)
    \]
  \item put ${F_1=E_1}$ and
    ${F_i=E_i\setminus \cup_1^{i-1} E_j}$ for ${i>1}$.  Then,
    \[
    \mu(\cup_1^\infty E_j)\stackrel{*}{=}
    \mu(\cup_1^\infty F_j)\stackrel{\ref{defn:measure:itm:2}}{=}
    \sum_1^\infty \mu(F_j)\stackrel{\ref{thm:01:itm:mon}}{\leq}
    \sum_1^\infty \mu(E_j)
    \]
  \item
    \[
    \mu(\cup_1^\infty E_j)\stackrel{\ref{defn:measure:itm:2}}{=}
    \sum_1^\infty \mu(E_j\setminus E_{j-1})=
    \lim_{n\rightarrow\infty} \mu(E_n)
    \]
  \item for ${j\in\mathbb{N}}$, put ${F_j=E_1\setminus E_j}$.  then,
    \[
    \forall j\in N,\forall x\in F_j,
    x\not\in E_j\land E_j\supset E_{j+1} \Rightarrow x\not\in E_{j+1}
    \text{ and }
    x\in E_j\text{ so } x\in F_{j+1}
    \]
    (i.e. -- ${F_1\subset F_2\subset\cdots}$).
    also, since ${E_j\subset E_1}$, we have
    ${E_1=E_j\sqcup F_j}$, and hence
    \begin{equation}
      \mu(E_1) = \mu(E_j)+\mu(F_j),\forall j\in N
      \tag{\dag}
    \end{equation}
    by the definition of measure \ref{defn:measure:itm:2}.  now,
    \begin{IEEEeqnarray*}{rl}
      & x\in\cup_1^\infty F_j = \cup_1^\infty (E_1\setminus E_j) \\
      \iff &
      x\in E_1\land\exists j\in N\st x\not\in E_j \\
      \iff &
      x\in E_1\land x\not\in\cap_1^\infty E_j
    \end{IEEEeqnarray*}
    (i.e. ---
    ${\cup_1^\infty F_j = E_1\setminus \cap_1^\infty E_j}$).
    Now since ${\cap_1^\infty E_j\subset E_1}$,
    ${E_1 = \cap_1^\infty E_j\sqcup \cup_1^\infty F_j}$,
    and by \ref{defn:measure:itm:2} and part
    \ref{thm:01:itm:cts-below},
    \[
    \mu(E_1)=\mu(\cap_1^\infty E_j)+\lim_{j\rightarrow\infty} \mu(F_j)
    \stackrel{\dag}{=} \mu(\cap_1^\infty E_j) + \lim_{j\rightarrow\infty} \big[
      \mu(E_1) - \mu(E_j)\big]
    \]
  \end{enumerate}
\end{proof}

\begin{rem}{}
  the condition ${\mu(E_1)<\infty}$ in part \ref{thm:01:itm:cts-above}
  of \ref{thm:01:measure-char} may be replaced by
  ${\exists n>1\st \mu(E_n)<\infty}$.
\end{rem}

\begin{defn}\
  \begin{itemize}
  \item given measure space ${(X, \mathcal{M}, \mu)}$,
    ${E\in \mathcal{M}\st \mu(E)=0}$ is called a \emph{null set}
    (or \emph{$\mu$-null set})
  \item a statement about points ${x\in X}$ that is true except
    possible for $x$ in some null set is said to be true
    \emph{almost everywhere} (abbreviated \emph{a.e.}) or for
    \emph{almost every} $x$ (or \emph{$\mu$-almost everywhere}).
  \end{itemize}
\end{defn}

\begin{rem}{}
  countable unions of null sets are null sets
\end{rem}

\begin{defn}
  if
  \[\mu(E)=0\land F\subset E\Rightarrow F\in M \]
  (i.e. the domain of $\mu$ includes all subsets of null sets),
  $\mu$ is a \emph{complete} measure
\end{defn}

\begin{thm}\label{thm:1.9}\label{thm:01:completion-existence}
  For any measure space ${(X, \mathcal{M}, \mu)}$, put
  \[ \mathcal{N} = \{N\in\mathcal{M}\mid\mu(N)=0\}, \]
  \[
  \overline{\mathcal{M}} = \{ E\cup F\mid
  E\in\mathcal{M}, F\subset N\text{ for }N\in\mathcal{N} \}.
  \]
  Then $\overline{\mathcal{M}}$ is a $\sigma$-algebra,
  and there's a unique extension $\overline\mu$ of $\mu$
  to a complete measure on $\overline{\mathcal{M}}$.
\end{thm}
\begin{proof}\
  \sketch{check $\overline{\mathcal{M}}$ a $\sigma$-algebra}
  Set ${\overline\mu(E\cup F) = \mu(E)}$
  \sketch{check well defined}
  \sketch{remaining left as $\circledast$Ex. 6$\circledast$}
\end{proof}

\begin{defn}
  $\overline\mu$ is the \emph{completion} of $\mu$,
  and $\overline{\mathcal{M}}$ is
  the \emph{completion} of $\mathcal{M}$ wrt. $\mu$.
\end{defn}

\section{Outer measures}
\begin{rem}{A}
  there is no need for a completely separate notion of
  ``inner measure'', since for a region, $E$, contained in
  a measurable set, $F$, we may take
  ${\mu(F)-\text{outer meas. }F\setminus E}$,
  or otherwise define in terms of outer measure
\end{rem}

\begin{defn}
  an \emph{outer measure} on a non-empty set, $X$, is a function
  ${\mu^*:\mathscr{P}(X)\rightarrow[0,\infty]}$ such that
  \begin{enumerate}[label=(\roman*)]
  \item\label{defn:outer-measure:itm:1}
    ${\mu^*(\emptyset)=0}$
  \item\label{defn:outer-measure:itm:2}
    ${A\subset B\Rightarrow\mu^*(A)\leq\mu^*(B)}$
  \item\label{defn:outer-measure:itm:subadd}
    ${\mu^*(\cup_1^\infty A_j)\leq\sum_1^\infty \mu^*(A_j)}$
  \end{enumerate}
\end{defn}

\begin{prop}\label{prop:1.10}\label{prop:01:induced-outer-measure}
  for any ${\mathcal{E}\subset\mathscr{P}(X)}$ and
  \[
  \rho:\mathcal{E}\rightarrow[0,\infty]
  \st
  \emptyset, X\in\mathcal{E} \land \rho(\emptyset) = 0
  \]
  \sketch{$\mathcal{E}$ for ``elementary''}
  put
  \[
  \mu^*(A) = \inf\left\{\sum_1^\infty \rho(E_j)\mid
  \{E_j\}\subset\mathcal{E} \land A\subset\cup_1^\infty E_j\right\}
  \text{ for } A\subset X.
  \]
  Then $\mu^*$ is an outer measure.
\end{prop}
\begin{proof}\
  \sketch{todo:
    \ref{defn:outer-measure:itm:1} and \ref{defn:outer-measure:itm:2}
  }
  \begin{itemize}
  \item[\ref{defn:outer-measure:itm:subadd}]
    fix ${\{A_j\}_1^\infty\subset\mathscr{P}(X)}$ and ${\epsilon>0}$.

    by definition of $\mu^*$, pick
    ${\{E_j^i\}_{i=1}^\infty\subset\mathcal{E}}$
    such that
    ${A_j\subset\cup_{i=1}^\infty E_j^i}$
    and
    \begin{equation}
    \sum_{i=1}^\infty \rho(E_j^i) \leq \mu^*(A_j)+\epsilon 2^{-j},
    \forall j\in \mathbb{N}
    \tag{\dag}
    \end{equation}

    then,
    ${\cup_1^\infty A_j\subset\cup_{i,j=1}^\infty E_j}$,
    so
    \[
    \mu^*(\cup_1^\infty A_j) \leq
    \sum_{i,j} \rho(E_j^i)
    \stackrel{\dag}{\leq}
    \sum_j \mu^*(A_j) + \epsilon
    \]
  \end{itemize}
\end{proof}

\begin{defn}
  for outer measure $\mu^*$ on $X$, ${A\subset X}$ is
  \emph{$\mu^*$-measurable} if
  \[
  \mu^*(E)=\mu^*(E\cap A) + \mu^*(E\cap \setcomp{A}),
  \forall E\subset X
  \]
\end{defn}

\begin{rem}{B}\label{rem:01:outer-measurable}\
  by \ref{defn:outer-measure:itm:subadd} (subadditivity),
  \[
  \mu^*(E)\leq \mu^*(E\cap A) +\mu^*(E\cap\setcomp{A}),
  \forall A, E\subset X
  \]
\end{rem}

\begin{rem}{C}
  intuitively, the idea of $\mu^*$-measurability is that
  ${\mu^*(A)}$ be equal to the ``inner measure'' of $A$,
  ${\mu^*(E) - \mu^*(E\cap\setcomp{A})}$.
\end{rem}

\begin{thm}[Carath\'eodory's]\label{thm:1.11}\label{thm:caratheodory}
  If $\mu^*$ is an outer measure on $X$,
  the $\mu^*$-measurable sets, $\mathcal{M}$, is a $\sigma$-algebra,
  and ${\restr{\mu^*}{\mathcal{M}}}$ is a complete measure.
\end{thm}
\begin{proof}
  ${A\in\mathcal{M}\Rightarrow\setcomp{A}\in\mathcal{M}}$
  by definition of $\mu^*$-measurable.
  For ${A,B\in\mathcal{M}}$ and ${E\subset X}$,
  \begin{IEEEeqnarray*}{Rl}
  \mu^*(E) = &
  \mu^*(E\cap A\cap B) +
  \mu^*(E\cap A\cap \setcomp{B}) \\
  + & \mu^*(E\cap \setcomp{A}\cap B) +
  \mu^*(E\cap \setcomp{A}\cap \setcomp{B})
  \end{IEEEeqnarray*}
  since ${A, B}$ $\mu^*$-measurable.
  Since
  \[
  A\cup B = (A\cap B)\cup(A\cap\setcomp{B})\cup(\setcomp{A}\cap B),
  \]
  subadditivity \ref{defn:outer-measure:itm:subadd} gives
  \[
  \mu^*\big(E\cap(A\cup B)\big) =
  \mu^*(E\cap A\cap B) + \mu^*(E\cap A\cap\setcomp{B}) +
  \mu^*(E\cap\setcomp{A}\cap B),
  \]
  so with the above equation for ${\mu^*(E)}$,
  \[
  \mu^*(E)\geq
  \mu^*\big(E\cap(A\cup B)\big) +
  \mu^*\big(E\cap\setcomp{(A\cup B)}\big),
  \]
  satisfying the (formally weaker) definition of
  $\mu^*$-measurable from \ref{rem:01:outer-measurable}
  for ${A\cup B}$.

  So $\mathcal{M}$ is an algebra of sets.

  \begin{IEEEeqnarray*}{rrCl}
    \IEEEeqnarraymulticol{2}{l}{
    \forall A,B\in\mathcal{M}, A\cap B =\emptyset \Rightarrow} &&\\
    & \mu^*(A\cup B) & \stackrel{A\in\mathcal{M}}{=} &
    \mu^*\big((A\cup B)\cap A\big)+
    \mu^*\big((A\cup B)\cap\setcomp{A}\big) \\
    && \stackrel{\cap=\emptyset}{=} &
    \mu^*(A)+\mu^*(B)
  \end{IEEEeqnarray*}
  (i.e. --- $\mu^*$ finitely additive on $\mathcal{M}$).

  Fix
  \begin{equation}\label{pf:thm:caratheodory:A_j-disjoint}
    \{A_j\}\subset\mathcal{M} \text{ disjoint}
    \tag{\dag}
  \end{equation}
  and put
  ${B_n=\cup_1^n A_j}$, ${B=\cup_1^\infty A_j}$.
  Then,
  \begin{IEEEeqnarray*}{LrCl}
    \forall E\subset X, & \mu^*(E\cap B_n) & = &
    \mu^*(E\cap B_n\cap A_n) + \mu^*(E\cap B_n\cap \setcomp{A_n}) \\ &
    & \stackrel{\ref{pf:thm:caratheodory:A_j-disjoint}}{=} &
    \mu^*(E\cap A_n) + \mu^*(E\cap B_{n-1}),
  \end{IEEEeqnarray*}
  so induct to find
  \begin{equation}\label{pf:thm:caratheodory:E-intersect-B_n}
    \tag{\ddag}
    \mu^*(E\cap B_n) = \sum_1^n \mu^*(E\cap A_j).
  \end{equation}

  Now,
  \begin{IEEEeqnarraytagged}{Rl}{\beth}\label{pf:thm:caratheodory:meas-E-1}
    \mu^*(E) \stackrel{\diamondsuit}{=} &
    \mu^*(E\cap B_n) + \mu^*(E\cap\setcomp{B_n}) \\
    \stackrel{\heartsuit}{\geq} &
    \mu^*(E\cap B_n) + \mu^*(E\cap\setcomp{B}) \\
    \stackrel{\ref{pf:thm:caratheodory:E-intersect-B_n}}{=} &
    \sum_1^n \mu^*(E\cap A_j) + \mu^*(E\cap\setcomp{B})
  \end{IEEEeqnarraytagged}
  or in the limit ${n\rightarrow\infty}$,
  \begin{IEEEeqnarraytagged}{Rl}{\star}\label{pf:thm:caratheodory:meas-E-2}
    \mu^*(E) \stackrel{\clubsuit}{\geq} &
    \mu^*\big(\cup_1^\infty (E\cap A_j)\big) + \mu^*(E\cap\setcomp{B}) \\
    \stackrel{\spadesuit}{=} &
    \mu^*(E\cap B) + \mu^*(E\cap\setcomp{B})
    \stackrel{\clubsuit}{\geq} \mu^*(E),
  \end{IEEEeqnarraytagged}
  where we have observed subadditivity ($\clubsuit$),
  set equality ($\spadesuit$),
  ${B_n\in\mathcal{M}}$ since $\mathcal{M}$ an algebra of sets
  ($\diamondsuit$), and
  property \ref{defn:outer-measure:itm:2} of outer measures
  together with
  ${B\supset B_n\Rightarrow\setcomp{B}\subset\setcomp{B_n}}$
  ($\heartsuit$).

  Squeezed between ${\mu^*(E)}$, the inequalities in
  (\ref{pf:thm:caratheodory:meas-E-2}) become equalities,
  so ${B\in\mathcal{M}}$, proving $\mathcal{M}$
  a $\sigma$-algebra by the formally weaker equivalent definition
  of $\sigma$-algebras from \ref{rem:01:sigma-algebra-equiv}
  in Section~\ref{sec:01:sigma-algebras}.  Setting
  ${E=B}$ in (\ref{pf:thm:caratheodory:meas-E-1})
  and squeezing equality in the limit ${n\rightarrow\infty}$
  as in (\ref{pf:thm:caratheodory:meas-E-2}) gives countable
  additivity on ${\mathcal{M}}$,
  \[ \mu^*(B) = \sum_1^\infty \mu^*(A_j). \]

  To show ${\restr{\mu^*}{\mathcal{M}}}$ complete,
  fix ${E, A\subset X}$ such that ${\mu^*(A) = 0}$.
  Then,
  \begin{IEEEeqnarray*}{rl}
    \mu^*(E)\stackrel{\clubsuit}{\leq} &
    \mu^*(E\cap A) + \mu^*(E\cap\setcomp{A}) \\
    \stackrel{\heartsuit}{=} &
    \mu^*(\setcomp{A}\cap E)
    \stackrel{\spadesuit}{\leq} \mu^*(E)
  \end{IEEEeqnarray*}
  implies ${A\in\mathcal{M}}$, by observing
  subadditivity ($\clubsuit$),
  property \ref{defn:outer-measure:itm:2} of outer measures
  ($\spadesuit$), and
  \begin{equation*}
    \tag{\heartsuit}
    E\cap A \subset A \Rightarrow \mu^*(E\cap A)\leq\mu^*(A)=0.
  \end{equation*}
  Hence $\mathcal{M}$ contains all sets of outer-measure zero
  and in particular all subsets of null sets.
\end{proof}

\begin{defn}
  given algebra of sets ${\mathcal{A}\subset\mathscr{P}(X)}$,
  a function ${\mu_0:\mathcal{A}\rightarrow[0,\infty]}$ is a
  \emph{premeasure} if
  \begin{enumerate}[label=(\roman*)]
  \item
    ${\mu_0(\emptyset)=0}$
  \item\label{defn:premeasure:itm:2}
    ${\{A_j\}_1^\infty\subset\mathcal{A}\text{ disjoint}\land \cup_1^\infty A_j\in\mathcal{A} \Rightarrow \mu_0(\cup_j A_j) = \sum_j\mu_0(A_j)}$
  \end{enumerate}
\end{defn}

\begin{rem}{D}
  \ref{defn:premeasure:itm:2} in the definition of premeasure
  implies finite additivity
\end{rem}

\begin{defn}
  \emph{finite} and \emph{$\sigma$-finite} premeasures
  defined just as for measures
\end{defn}

\begin{rem}{E}\label{rem:01:premeasure2outer-measure}
  by \ref{prop:01:induced-outer-measure},
  \[
  \mu^*(E) = \inf\left\{\sum_1^\infty \mu_0(A_j)\mid
  \{A_j\}\subset\mathcal{A} \land E\subset\cup_1^\infty A_j\right\}
  \]
  is an outer measure on $X$ for any premeasure $\mu_0$
  on any algebra ${\mathcal{A}\subset\mathscr{P}(X)}$.
\end{rem}

\begin{prop}\label{prop:1.13}\label{prop:01:premeasure-measurable}
  for $\mu_0$, $\mu^*$, $\mathcal{A}$
  as in \ref{rem:01:premeasure2outer-measure},
  \begin{enumerate}[label=(\alph*)]
  \item\label{prop:01:itm:premeasure-eq-outer-measure}
    ${\restr{\mu^*}{\mathcal{A}} = \mu_0}$
  \item\label{prop:01:itm:premeasure-measurable}
    elements of $\mathcal{A}$ are $\mu^*$-measurable
  \end{enumerate}
\end{prop}
\begin{proof}\
  \begin{enumerate}[label=(\alph*)]
  \item
    fix ${E\in\mathcal{A}}$ and
    ${\{A_j\}\subset\mathcal{A}}$ such that
    ${E\subset \cup_1^\infty A_j}$, and put
    \[ B_n=E\cap\big(A_n \setminus \cup_1^{n-1} A_j\big). \]
    Then
    \[
    \{B_j\}_1^\infty\subset\mathcal{A}\text{ disjoint}
    \land
    \cup B_j = E \in \mathcal{A}
    \Rightarrow
    \mu_0(E) = \sum_1^\infty \mu_0(B_j)
    \]
    by \ref{defn:premeasure:itm:2}.  So
    ${\mu_0(E)\leq\sum_1^\infty \mu_0(A_j)}$,
    because (the proof of) monotonicity
    (\ref{thm:01:measure-char} \ref{thm:01:itm:mon})
    holds for premeasures as well as measures, and hence
    ${\mu_0(E)\leq\mu^*(E)}$ by definition of $\mu^*$.
  \item
    Fix ${A\in\mathcal{A}}$, ${E\subset X}$, ${\epsilon>0}$, and
    by definition of $\mu^*$, choose
    \[
    \{B_j\}_1^\infty \subset\mathcal{A}
    \st E\subset\cup_1^\infty B_j \land
    \sum_1^\infty \mu_0(B_j)\leq \mu^*(E) + \epsilon .
    \]
    Then,
    \begin{IEEEeqnarray*}{Rl}
    \mu^*(E) \stackrel{\dag}{\geq} &
    \sum_1^\infty \mu_0(B_j\cap A) + \sum_1^\infty \mu_0(B_j\setcomp{A}) \\
    \stackrel{\ddag}{\geq} &
    \mu_0(E\cap A) + \mu_0(E\cap\setcomp{A})
    \end{IEEEeqnarray*}
    \sketch{spell out $\dag$ and $\ddag$}
  \end{enumerate}
\end{proof}

% todo: more symbolic typesetting like notes in statement
\begin{thm}\label{thm:1.14}\label{thm:premeasure-to-measure}
  Let ${\mathcal{A}\subset\mathscr{P}(X)}$ be an algebra of sets,
  $\mu_0$ a premeasure on $\mathcal{A}$,
  $\mathcal{M}$ the $\sigma$-algebra generated by $\mathcal{A}$.
  Then,
  \begin{enumerate}
  \item\label{thm:premeasure-to-measure:itm:1}
    ${\mu = \restr{\mu^*}{\mathcal{M}}}$ is a measure on $\mathcal{M}$
    such that ${\restr{\mu}{\mathcal{A}}=\mu_0}$,
    where $\mu^*$ is defined as in \ref{rem:01:premeasure2outer-measure}
  \item\label{thm:premeasure-to-measure:itm:2}
    for any measure $\nu$ on $\mathcal{M}$,
    ${\restr{\nu}{\mathcal{A}}=\mu_0}$ implies both
    \[ \nu(E)\leq\mu(E),\,\forall E\in\mathcal{M} \]
    and
    \[ \mu(E)\Rightarrow \nu(E)=\mu(E) \]
  \item\label{thm:premeasure-to-measure:itm:3}
    $\mu_0$ $\sigma$-finite implies
    $\mu$ the unique extension of $\mu_0$ to a measure on $\mathcal{M}$.
  \end{enumerate}
\end{thm}
\begin{proof}\
  \begin{itemize}
  \item[\ref{thm:premeasure-to-measure:itm:1}]
    Since $\mu^*$ is an outer measure by
    Remark~\ref{rem:01:premeasure2outer-measure},
    Carath\'eodory's theorem, \ref{thm:caratheodory}, implies
    $\widetilde{\mu} = \restr{\mu^*}{\widetilde{\mathcal{M}}}$
    is a measure on $\widetilde{\mathcal{M}}$, the $\sigma$-algebra
    of $\mu^*$-measurable sets.
    And since $\mathcal A\subset \widetilde{\mathcal{M}}$,
    \ref{prop:01:premeasure-measurable} part
    \ref{prop:01:itm:premeasure-measurable} implies
    $\mathcal M = \mathcal M(\mathcal A) \subset \widetilde{\mathcal{M}}$
    by definition of generated $\sigma$-algebra.
    Hence ${\mu = \restr{\widetilde{\mu}}{\mathcal M}}$
    is a measure, since the restriction of a measure to a
    $\sigma$-algebra is again a measure.
    Finally, since $\mathcal A\subset \mathcal M$,
    \ref{prop:01:premeasure-measurable} part
    \ref{prop:01:itm:premeasure-eq-outer-measure} provides
    $\restr{\mu}{\mathcal A} = \restr{\mu^*}{\mathcal A} = \mu_0$.
  \item[\ref{thm:premeasure-to-measure:itm:2}]
    Fix $\nu$ satsifying hypothesis and $E\in\mathcal M$. Then,
    ${\forall \{A_j\}_1^\infty\subset\mathcal A \st
      E\subset\cup_1^\infty A_j}$
    \[
    \nu(E) \stackrel{\spadesuit}{\leq} \sum_1^\infty \nu(A_j)
    \stackrel{\clubsuit}{=} \sum_1^\infty \mu_0(A_j)
    \]
    by monotonicity and subadditivity of measures ($\spadesuit$)
    and choice of $\nu$ ($\clubsuit$), proving
    $\nu(E)\leq\mu(E)$ by definition of $\mu$ and $\mu^*$.
    And setting $A=\cup_1^\infty A_j$,
    \begin{equation*}
      \tag{*}
      \nu(A) \stackrel{
        \ref{thm:01:measure-char}\ref{thm:01:itm:cts-below}}{
        =} \lim_{n\rightarrow\infty}\nu(\cup_1^n A_j)
      \stackrel{\dag}{=} \lim_{n\rightarrow\infty}\nu(\cup_1^n A_j)
      \stackrel{
        \ref{thm:01:measure-char}\ref{thm:01:itm:cts-below}}{
        =} \mu(A)
    \end{equation*}
    \begin{equation*}
      \tag{\dag}
      \restr{\nu}{\mathcal A} = \mu_0 \land
      \restr{\mu}{\mathcal A\subset\mathcal M}
      = \restr{\mu^*}{\mathcal A} = \mu_0 \land
      \mathcal{A}\text{ closed under finite unions}.
    \end{equation*}
    \paragraph{}
    If $\mu(E)<\infty$,
    by definition of $\mu$ and $\mu^*$,
    \[
    \forall \epsilon>0 \exists \{A_j\}_1^\infty\subset\mathcal A \st
    E\subset \cup_1^\infty A_j =: A \land
    \big [
      \mu(A) < \mu(E) + \epsilon \Leftrightarrow
      \mu(A\setminus E) < \epsilon
      \big ]
    \]
    and hence
    \[
    \mu(E)\leq\mu(A)\stackrel{*}{=} \nu(A)
    = \nu(E) + \nu(A\setminus E)
    \stackrel{\ddag}{\leq} \nu(E) + \mu(A\setminus E)
    \leq \nu(E) + \epsilon,
    \]
    where $\ddag$ follows from the first statement of this part as
    proven in the previous paragraph, completing remaining inequality
    for the proof that $\mu(E)=\nu(E)$ in the case that $\mu(E)<\infty$
    since $\epsilon>0$ arbitrary.
  \item[\ref{thm:premeasure-to-measure:itm:3}]
    By hypothesis, choose
    \[
    \{B_j\}_1^\infty\subset\mathcal A \st
    \bigcup_1^\infty B_j = X \land
    \mu_0(B_j)<\infty,\forall j\in\mathbb N.
    \]
    Set $A_1 = B_1$ and $A_{j+1} = B_{j+1}\setminus\cup_1^j B_j$
    for $j>1$ such that
    \[
    \{A_j\}_1^\infty\subset\mathcal A \land
    \bigsqcup_1^\infty A_j = X \land
    \mu_0(A_j)<\infty,\forall j\in\mathbb N
    \]
    (i.e. --- choose the witness to $\sigma$-finiteness such that
    it's made of disjoint elements).
    \sketch{todo: double-check while extracting as lemma}
    \paragraph{}
    Then, for any measure $\nu:\mathcal M\rightarrow [0,\infty]$
    extending $\mu_0$,
    \[
    \mu(E) \stackrel{\dag}{=} \sum_1^\infty \mu(E\cap A_j)
    \stackrel{\ddag}{=}
    \sum_1^\infty \nu(E \cap A_j) \stackrel{\dag}{=} \nu(E)
    \]
    by countable additivity ($\dag$) as well as
    monotonicity and the previous two parts ($\ddag$).
  \end{itemize}
\end{proof}

\section{Borel measures on the real line}
\begin{defn}
  \emph{h-intervals} are the half-open intervals
  \[
  \{(a,b]\mid a\in\overline R\setminus\{\infty\}, b\in R, a<b\} \cup
    \{(a,\infty)\mid a\in\overline R\} \cup \{\emptyset\} \subset R
    \]
\end{defn}

\begin{rem}{}
  let $H$ denote the set of h-intervals and $\mathcal A$ denote
  the set of finite disjoint unions of h-intervals.
  \begin{itemize}
    \item
      h-intervals are an elementary family, so $\mathcal A$
      is an algebra by \ref{prop:01:elemfamalg}
    \item
      \[
      H \subset \mathcal A \stackrel{\ref{lem:01:gensa}}{\Rightarrow}
      \mathcal M(H) \stackrel{\ref{prop:01:borelrealsgen}}{=}
      \mathcal B_{\mathbb R} \subset \mathcal M(\mathcal A),
      \]
      and
      \[
      \mathcal A\subset \mathcal M(H) = \mathcal B_{\mathbb R}
      \stackrel{\dag}{\Rightarrow}
      \mathcal M(\mathcal A)\subset \mathcal B_{\mathbb R}
      \]
      \begin{equation*}
        \tag{\dag}
        \mathcal M(\mathcal B) = \mathcal B,
        \forall \sigma\text{-algebra } \mathcal B
      \end{equation*}
  \end{itemize}
\end{rem}

\begin{defn}
  \begin{itemize}
  \item
    \emph{Borel measures} on $R$ are measures with the domain
    $\mathcal B_R$, the Borel $\sigma$-algebra on $R$
  \item
    given a Borel measure, $\mu$, on $R$,
    \[F: R\rightarrow [0,\infty]\quad x\mapsto\mu\big((-\infty,x]\big)\]
      is the \emph{distribution function} of $\mu$
  \end{itemize}
\end{defn}

\begin{rem}{}
  distribution functions are increasing by monotonicity of measure
  and right continuous by continuity from above of measure
  (theorem~\ref{thm:01:measure-char})
\end{rem}
\sketch{todo: check proof, particularly of right cts.
  and provide counterexample for left cts. (point measure)}

\begin{prop}\label{prop:1.15}
  Let $H$ denote the set of h-intervals and $\mathcal A$ denote
  the set of finite disjoint unions of h-intervals.

  For increasing, right-continuous $F: R\rightarrow R$,
  the map
  \[
  \mu_0:\mathcal A\rightarrow [0,\infty]
  \qquad \bigsqcup_1^n (a_j, b_j] \mapsto \sum_1^n [F(b_j)-F(a_j)]
    \quad \emptyset \mapsto 0 \quad (a, \infty)\mapsto\infty
    \]
    is a premeasure on $\mathcal A$.
\end{prop}
\begin{proof}
  For finite, disjoint $\{I_i\}_{i=1}^n, \{J_j\}_{j=1}^m\subset H$
  such that $\cup_i I_i = \cup_j J_j$,
  \[
  \sum_i \mu_0(I_i) = \sum_{i,j} \mu_0(I_i\cap J_j) = \sum_j \mu_0(J_j).
  \]
  Indeed, for any $i\in[n]$,
  \[
  I_i \subset \bigcup_{j=1}^m J_j \land \{J_j\}_{j=1}^m \text{ disjoint}
  \Rightarrow I_i = \bigsqcup_{j=1}^m (I_i\cap J_j)
  \]
  and $I_i\cap J_j \in H, \forall (i,j)\in[n]\times[m]$ since $H$
  an elementary family imply
  ${\mu_0(I_i) = \sum_{j=1}^m \mu_0(I_i\cap J_j)}$.
  Similarly,
  $\forall j\in[m],\mu_0(J_j)=\sum_{i=1}^n \mu_0(I_i\cap J_j)$.
  So $\mu_0$ is well-defined.
  \paragraph{}
  Fix disjoint
  $\{A_j\}_1^\infty\subset\mathcal A \st \cup_1^\infty A_j\in\mathcal A$.
  Now
  \[\exists\text{ disjoint }\{I_j\}_1^\infty\subset H \st
  \bigcup_1^\infty I_j = \bigcup_1^\infty A_j \]
  definition of $\mathcal A$ since the countable union of finite sets
  is countable, so by definition of $\mu_0$, proving
  \ref{defn:premeasure:itm:2} reduces to proving
  \[
  \{I_j\}_1^\infty \subset H \text{ disjoint }
  \land \cup_1^\infty I_j \in\mathcal A \Rightarrow
  \mu_0\left(\cup_1^\infty I_j\right)=\sum_1^\infty\mu_0(I_j).
  \]
  Or, WLoG, by definitions of $\mathcal A$ and $\mu_0$, to proving
  \begin{equation}
    \tag{*}
    \{I_j\}_1^\infty \subset H \text{ disjoint }
    \land I:=\cup_1^\infty I_j=:(a,b] \in H \Rightarrow
      \mu_0\left(\cup_1^\infty I_j\right)=\sum_1^\infty\mu_0(I_j).
  \end{equation}
  \sketch{Partition original choice of $\{I_j\}$ into finitely many
  subsequences according to which of the finitely many h-intervals in
  $\cup_1^\infty I_j$ contains each $I_j$...
  ... unclear if making this symbolic will clarify or add clutter}
  \paragraph{}
  Fix $\{I_j\}_1^\infty \subset H$ satisfying hypothesis of ($*$).
  Now, for any $n\in\mathbb N$,
  \[
  \mu_0(I) \stackrel{\dag}{=}
  \mu_0(\cup_1^n I_j) + \mu_0(I\setminus\cup_1^n I_j)
  \stackrel{\heartsuit}{\geq} \mu_0(\cup_1^n I_j)
  \stackrel{\ddag}{=} \sum_1^n \mu_0(I_j)
  \]
  where
  \begin{itemize}
  \item
    $\dag$ follows from definition of $\mu_0$ since
    removing an h-interval (and hence any finite union of h-intervals)
    from a finite disjoint union of h-intervals is again
    a finite disjoint union of h-intervals,
  \item
    $\heartsuit$ is due to the observation that $\mu_0$
    is non-negative, and
  \item
    $\ddag$ is a direct application of the definition of $\mu_0$.
  \end{itemize}
  Since this inequality bounds a monotonically increasing sequence
  from above, the sequence converges, and
  \[ \mu_0(I)\geq \sum_1^\infty \mu_0(I_j). \]
  \paragraph{}
  For the reverse inequality in the case $a, b\neq\infty$,
  fix $\epsilon > 0$.
  Since $F$ right-continuous, choose $\delta>0$ and
  $\{\delta_j > 0\}_1^\infty$ such that
  \[
  F(a+\delta) - F(a) < \epsilon
  \quad\text{and}\quad
  F(b_j+\delta_j)-F(b_j) < \epsilon 2^{-j},\forall j\in\mathbb N,
  \]
  where $(a_j, b_j] := I_j$, shrinking $\delta$ such that
  $a + \delta<b$ if necessary.
  Now
  \[
  [a+\delta, b]\subset (a, b] = \cup_1^\infty (a_j, b_j]
  \subset \cup_1^\infty (a_j, b_j + \delta_j)
  \]
  and $[a+\delta, b]$ compact provides finite subcover
  of $\cup_1^\infty (a_j, b_j + \delta_j)$, so reorder the cover,
  possibly discarding some elements in case
  ${(a_\ell,b_\ell+\delta_\ell)\subset(a_k,b_k+\delta_k)}$,
  and choose $N\in\mathbb N$ such that
  \begin{equation*}
    \tag{\diamondsuit}
  [a+\delta, b]\subset\cup_1^N (a_j, b_j+\delta_j) \land
  b_j+\delta_j\in(a_{j+1},b_{j+1}+\delta_{j+1}), \forall j\in[N-1].
  \end{equation*}
  Then,
  \begin{IEEEeqnarray*}{rl}
    & \mu_0(I) = F(b) - F(a) < F(b) - F(a+\delta) +\epsilon
    \stackrel{\dag}{\leq} F(b_N+\delta_N) - F(a_1) + \epsilon
    \\
    = &
    F(b_N+\delta_N) - F(a_N) +
    \sum_1^{N+1}\big[F(a_{j+1})-F(a_j)\big]+\epsilon \leq
    \\
    \stackrel{\ddag}{\leq} &
    F(b_N+\delta_N) - F(a_N) +
    \sum_1^{N+1}\big[F(b_j + \delta_j)-F(a_j)\big]+\epsilon < \\
    < & \sum_1^N\big[F(b_j) \epsilon2^{-j}-F(a_j)\big] + \epsilon
    < \sum_1^\infty\mu_0(I_j)+2\epsilon
  \end{IEEEeqnarray*}
  since
  \begin{itemize}
  \item[$\dag$]
    $a_1<a+\delta\leq b<b_N+\delta_N$ by
    $\diamondsuit$ and choice of $\delta$
  \item[$\ddag$]
    $a_{j+1} < b_j + \delta_j$ by $\diamondsuit$
  \end{itemize}
  \paragraph{}
  In the case $a=-\infty$, consider finite covers of $[-M, b]$
  for $M<\infty$ and similarly find
  \[
  \forall M<\infty,\forall\epsilon>0\st -M<b,
  F(b)-F(-M)\leq \sum_1^\infty \mu_0(I_j) + 2\epsilon.
  \]
  And in the case $b=\infty$, similary
  \[
  F(M)-F(a)\leq \sum_1^\infty \mu_0(I_j) + 2\epsilon,
  \forall M<\infty,\forall\epsilon>0\st M>a.
  \]
  So take the limit of both sides as
  $\epsilon\rightarrow\infty$ and $M\rightarrow\infty$
  in both cases to prove because $M,\epsilon$ quantified over
  each statement.
  \sketch{in case $I=R$, partition $\{I_j\}$ about some point
    and apply the previous two cases}
\end{proof}
